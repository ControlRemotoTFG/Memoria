%---------------------------------------------------------------------
%
%                      agradecimientos.tex
%
%---------------------------------------------------------------------
%
% agradecimientos.tex
% Copyright 2009 Marco Antonio Gomez-Martin, Pedro Pablo Gomez-Martin
%
% This file belongs to the TeXiS manual, a LaTeX template for writting
% Thesis and other documents. The complete last TeXiS package can
% be obtained from http://gaia.fdi.ucm.es/projects/texis/
%
% Although the TeXiS template itself is distributed under the 
% conditions of the LaTeX Project Public License
% (http://www.latex-project.org/lppl.txt), the manual content
% uses the CC-BY-SA license that stays that you are free:
%
%    - to share & to copy, distribute and transmit the work
%    - to remix and to adapt the work
%
% under the following conditions:
%
%    - Attribution: you must attribute the work in the manner
%      specified by the author or licensor (but not in any way that
%      suggests that they endorse you or your use of the work).
%    - Share Alike: if you alter, transform, or build upon this
%      work, you may distribute the resulting work only under the
%      same, similar or a compatible license.
%
% The complete license is available in
% http://creativecommons.org/licenses/by-sa/3.0/legalcode
%
%---------------------------------------------------------------------
%
% Contiene la p�gina de agradecimientos.
%
% Se crea como un cap�tulo sin numeraci�n.
%
%---------------------------------------------------------------------

\chapter{Agradecimientos}

\cabeceraEspecial{Agradecimientos}

\begin{FraseCelebre}
\begin{Frase}
Nadie es innecesario.
\end{Frase}
\begin{Fuente}
Yit\'an, Final Fantasy IX
\end{Fuente}
\end{FraseCelebre}

El primer agradecimiento hay que darselo a la Universidad Complutense por aceptar la creaci\'on de este grado, un grado que demuestra la importancia del mundo de los videojuegos en la sociedad actual. Con este grado se han conseguido romper muchas barreras, entre ellas est\'a poder especializarse y adoptar los videojuegos como nuestra profesi\'on, y tras 4 a\~nos podemos decir que no ha sido f\'acil, pero somos ingenieros y desarrolladores de videojuegos.
\\
\\
Dar gracias a los profesores que nos han acompa\~nado estos a\~nos y que han contribuido en el desarrollo del grado. Una menci\'on aparte para las dos personas que han hecho posible la realizaci\'on de este Trabajo de Fin de Grado, Carlos Le\'on Aznar y Pedro Pablo G\'omez Mart\'in.
\\
\\
Nuestro \'ultimo  agradecimiento va dirigido a nuestras familias. No ha sido f\'acil aguantar las noches de desvelo, alegr\'ias y enfados tras estos 4 a\~nos en la universidad.
\\

\endinput
% Variable local para emacs, para  que encuentre el fichero maestro de
% compilaci�n y funcionen mejor algunas teclas r�pidas de AucTeX
%%%
%%% Local Variables:
%%% mode: latex
%%% TeX-master: "../Tesis.tex"
%%% End:
