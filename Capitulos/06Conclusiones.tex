
\chapter{Conclusiones}
\label{cap7}
\label{cap:conclusiones}

El objetivo del proyecto es dise\~nar e implementar una herramienta con la que poder utilizar un dispositivo m\'ovil como dispositivo de entrada para videojuegos. Para probar que esta herramienta funciona, se ha implementado la herramienta desarrollada en un proyecto ya finalizado de Unity. La integraci\'on de la herramienta en esta demo ha requerido de la implementaci\'on de scripts auxiliares y modificaciones espec\'ificas del proyecto desarrollado por Unity. Las pruebas realizadas con usuarios muestran unos datos positivos ya que todas las ejecuciones han podido realizarse sin inconvenientes a pesar de la situaci\'on actual. Esto demuestra que la herramienta se comporta de manera favorable en entornos diferentes y con configuraciones no preestablecidas. Esta situaci\'on ha reflejado que existe un problema cuando las dimensiones de los dispositivos Android no son las mismas que las que se pens\'o inicialmente. 

Como puede verse en los resultados de las pruebas, gran parte de la fluidez depende del procesador del ordenador donde se ejecute el juego. Esto indica que el proceso de compresi\'on de imagen es muy lento en algunas de las pruebas realizadas. La latencia de red y la velocidad de descompresi\'on de la imagen en Android son muy \'optimas debido a que los dispositivos m\'oviles utilizados en las pruebas eran de gama media-alta. 


%-------------------------------------------------------------------
\section{Trabajo Futuro}
%-------------------------------------------------------------------

Como se ha comprobado, el proyecto y la herramienta cumplen los objetivos definidos en el cap\'itulo 1 y 3. Sin embargo, debido a los problemas para realizar las pruebas con usuarios en un entorno controlado se han descubierto una serie de fallos. A continuaci\'on se proporcionan una serie de tareas a realizar en una posible revisi\'on de la herramienta:

\begin {itemize}
\item Mejorar los tiempos de descompresi\'on de imagen en sistemas m\'oviles de gamas m\'as bajas.
\item Mejorar los tiempos de compresi\'on de textura a PNG en ordenadores con procesadores de gamas bajas.
\item Eliminar la restricci\'on que obliga a ambos dispositivos a estar conectados a la misma red.
\item Tener en cuenta las dimensiones del dispositivo m\'ovil para modificar la posici\'on de los botones.
\item A\~nadir una lista de posibilidades para elegir con qu\'e mando se desea jugar.
\end {itemize}