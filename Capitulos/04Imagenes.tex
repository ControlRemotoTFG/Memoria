%---------------------------------------------------------------------
%
%                          Cap�tulo 4
%
%---------------------------------------------------------------------
%
% 04Imagenes.tex
% Copyright 2009 Marco Antonio Gomez-Martin, Pedro Pablo Gomez-Martin
%
% This file belongs to the TeXiS manual, a LaTeX template for writting
% Thesis and other documents. The complete last TeXiS package can
% be obtained from http://gaia.fdi.ucm.es/projects/texis/
%
% Although the TeXiS template itself is distributed under the 
% conditions of the LaTeX Project Public License
% (http://www.latex-project.org/lppl.txt), the manual content
% uses the CC-BY-SA license that stays that you are free:
%
%    - to share & to copy, distribute and transmit the work
%    - to remix and to adapt the work
%
% under the following conditions:
%
%    - Attribution: you must attribute the work in the manner
%      specified by the author or licensor (but not in any way that
%      suggests that they endorse you or your use of the work).
%    - Share Alike: if you alter, transform, or build upon this
%      work, you may distribute the resulting work only under the
%      same, similar or a compatible license.
%
% The complete license is available in
% http://creativecommons.org/licenses/by-sa/3.0/legalcode
%
%---------------------------------------------------------------------

\chapter{Arquitectura}
\label{cap4}
\label{cap:arquitectura}


\begin{FraseCelebre}
\begin{Frase}
%El alma nunca piensa sin una imagen mental.
\end{Frase}
\begin{Fuente}
%Arist�teles
\end{Fuente}
\end{FraseCelebre}

\begin{resumen}

\end{resumen}

%-------------------------------------------------------------------
\section{Arquitectura General}
%-------------------------------------------------------------------
\label{cap4:sec:arquitectura_general}
Desde el punto del desarrolador, para poder utilizar esta herramienta va a ser necesario tener la versi\'on XX.XX de Unity3D y Android Studio en su versi\'on XX.XX y un tel\'efono m\'ovil que podamos utilizar como emulador.
\\
Esta herramienta tiene 2 partes que est\'an diferenciadas por la plataforma para las que van destinadas. En el lado de Unity3D, se ha implementado una serie de clases que son ajenas al motor donde se est\'a ejecutando el juego. 
Estas clases est\'an escritas en C\# y utilizan las librer\'ias  nativas de red de este lenguaje para comunicaci\'on. Por encima de estas clases se encuentra la parte espec\'ifica de Unity3D que facilita su uso. 
\\
Como parte extra a la herramienta, se ha propuesto un inicio de conexi\'on a trav\'es de la lectura de un QR. Con esto lo que se pretende es enlazar el dispositivo Android con el juego que se est\'a ejecutando en Unity. 
La lectura de este QR se hace directamente desde la aplicaci\'on creada para este proyecto. Al leer el QR y emparejarse ambos dispositivos, la aplicaci\'on Android cambia de c\'amara a GamePad.
\\

[INSERTE IMAGEN DE ARQUITECTURA Y DIAGRAMA DE CLASES Y COMPONENTES  + EXPLICACION]


%-------------------------------------------------------------------
\section{Protocolo de conexi\'on}
%-------------------------------------------------------------------
\label{cap4:sec:protocolo}

El protocolo se incia con la lectura de un QR que se mostrar\'a en el juego ejecutado en Unity. Una vez el dispositivo Android haya leido ese QR, Android manda la resoluci\'on del dispositivo con 2 bytes. 
El servidor que se est\'a ejecutando en Unity, guarda esa informaci\'on y escala el mando que se haya seleccionado a la resoluci\'on que tiene el dispositivo Android. Una vez hecho esto, se calculan las coordenadas de los botones del mando y se pasan a un XML.
\\
Este XML junto a la imagen es enviado al dispositivo Android. El tel\'efono guarda esta informaci\'on y deja de mostrar la c\'amara para mostrar la imagen del mando que se le ha enviado. En la aplicaci\'on Android han quedado guardadas las coordenadas de los botones, por lo que ya podr\'ia empezar la sesi\'on de juego.
\\
Durante una sesi\'on de juego estandar, el dispositivo Android recibir\'a pulsaciones del jugador en la pantalla, calcular\'a el bot\'on al que corresponde y enviar\'a al ordenador un array de 10 bytes en los que ir\'a la informaci\'on necesaria para saber la acci\'on que ha hecho el jugador y reaccionar consecuentemente.
\\

[ADJUNTAR IMAGEN DESCRIPTIVA DEL CONTENIDO DEL ARRAY + POSTERIOR EXPLICACION DEL SIGNIFICADO DE CADA COSA Y LOS VALORES QUE PUEDE TOMAR]

%-------------------------------------------------------------------
\section{El uso de los QR para inicio de conexi\'on}
%-------------------------------------------------------------------
\label{cap4:QR}



%-------------------------------------------------------------------
\section{Pantalla secundaria y controlador}
%-------------------------------------------------------------------
\label{cap4:pantallasecundaria}





% Variable local para emacs, para  que encuentre el fichero maestro de
% compilaci�n y funcionen mejor algunas teclas r�pidas de AucTeX
%%%
%%% Local Variables:
%%% mode: latex
%%% TeX-master: "../ManualTeXiS.tex"
%%% End:
