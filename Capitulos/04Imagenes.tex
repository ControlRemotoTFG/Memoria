%---------------------------------------------------------------------
%
%                          Cap�tulo 4
%
%---------------------------------------------------------------------
%
% 04Imagenes.tex
% Copyright 2009 Marco Antonio Gomez-Martin, Pedro Pablo Gomez-Martin
%
% This file belongs to the TeXiS manual, a LaTeX template for writting
% Thesis and other documents. The complete last TeXiS package can
% be obtained from http://gaia.fdi.ucm.es/projects/texis/
%
% Although the TeXiS template itself is distributed under the 
% conditions of the LaTeX Project Public License
% (http://www.latex-project.org/lppl.txt), the manual content
% uses the CC-BY-SA license that stays that you are free:
%
%    - to share & to copy, distribute and transmit the work
%    - to remix and to adapt the work
%
% under the following conditions:
%
%    - Attribution: you must attribute the work in the manner
%      specified by the author or licensor (but not in any way that
%      suggests that they endorse you or your use of the work).
%    - Share Alike: if you alter, transform, or build upon this
%      work, you may distribute the resulting work only under the
%      same, similar or a compatible license.
%
% The complete license is available in
% http://creativecommons.org/licenses/by-sa/3.0/legalcode
%
%---------------------------------------------------------------------

\chapter{Especificaci\'on}
\label{cap4}
\label{cap:especificacion}

%-------------------------------------------------------------------
\section{Introducci\'on}
%-------------------------------------------------------------------
\label{cap4:sec:intro}

Con este proyecto lo que se busca es convertir a un dispositivo m\'ovil en un dispositivo de entrada para jugar a videojuegos. Para conseguir esto se necesita que al menos 2 dispositivos se comuniquen entre si, en este caso un ordenador y un tel\'efono Android.
En el ordenador lo que se quiere es ejecutar un videojuego y que este se controle atrav\'es de una aplicaci\'on m\'ovil que se conecte y transfiera datos de pulsaciones y acciones del jugador para que pueda jugar al juego desde su tel\'elefono m\'ovil. Para que esta comunicaci\'on sea posible, es necesario un protocolo de comunicaci\'on. Con este protocolo de cmunicaci\'on ambas aplicaciones tendr\'an un control sobre los datos que envian y reciben para poder actuar en consecuencia.

%-------------------------------------------------------------------
\section{Funcionalidad}
%-------------------------------------------------------------------

\begin {itemize}
\item Env\'io de pulsaciones y gestos admisibles en pantallas t\'actiles.
\item Env\'io de im\'agenes de manera constante. Streaming de video.
\item Vubraci\'on y el uso de vibraci\'on como feedback.
\item ?`Nombrar aqui posible trabajo futuro y ya desarrollarlo en el cap\'itulo que corresponde?
\end {itemize}

%-------------------------------------------------------------------
\section{Protocolo de comunicaci\'on}
%-------------------------------------------------------------------
\begin {itemize}
\item Diagrama del protocolo con explicaci\'on exhaustiva sobre el uso del little o big endian (coger por ejemplo el protocolo HTTP para ver como se realiza una especificaci\'on detallada)
\item Explicaci\'on de cada uno de los mensajes que se env\'ian y reciben. Hab\'ia pensado poner los cl\'asicos diagramas de cajas para ver la estructura de los bytes que se envian y el orden sobretodo para una posterior lectura si alguien utiliza solamente una de las partes del proyecto.
\end {itemize}

% Variable local para emacs, para  que encuentre el fichero maestro de
% compilaci�n y funcionen mejor algunas teclas r�pidas de AucTeX
%%%
%%% Local Variables:
%%% mode: latex
%%% TeX-master: "../ManualTeXiS.tex"
%%% End:
