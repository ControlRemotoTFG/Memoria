%---------------------------------------------------------------------
%
%                          Cap�tulo 4
%
%---------------------------------------------------------------------
%
% 04Imagenes.tex
% Copyright 2009 Marco Antonio Gomez-Martin, Pedro Pablo Gomez-Martin
%
% This file belongs to the TeXiS manual, a LaTeX template for writting
% Thesis and other documents. The complete last TeXiS package can
% be obtained from http://gaia.fdi.ucm.es/projects/texis/
%
% Although the TeXiS template itself is distributed under the 
% conditions of the LaTeX Project Public License
% (http://www.latex-project.org/lppl.txt), the manual content
% uses the CC-BY-SA license that stays that you are free:
%
%    - to share & to copy, distribute and transmit the work
%    - to remix and to adapt the work
%
% under the following conditions:
%
%    - Attribution: you must attribute the work in the manner
%      specified by the author or licensor (but not in any way that
%      suggests that they endorse you or your use of the work).
%    - Share Alike: if you alter, transform, or build upon this
%      work, you may distribute the resulting work only under the
%      same, similar or a compatible license.
%
% The complete license is available in
% http://creativecommons.org/licenses/by-sa/3.0/legalcode
%
%---------------------------------------------------------------------

\chapter{Implementaci\'on de las aplicaciones}
\label{cap4}
\label{cap:impl}

%-------------------------------------------------------------------
\section{Android}
%-------------------------------------------------------------------

\begin {itemize}
\item Explicaci\'on muy despacito de qu\'e es un ciclo de vida. Hablar sobre Actividades y los estados que tienen estas Actividades.
\item Explciaci\'on de niveles de API y permisos. La aplicaci\'on est\'a preparada para leer un c\'odigo QR si o si en el formato IP:Puerto. Todo esto hace que mencione el QR al hablar de Android, seria recomendable hacer una subseccion para hablar del uso del QR, "historia" de este tipo de c\'odigos, etc?
\item Veis mejor hacer aqui una subseccion para explicar arquitectura y otra para explicacion del uso?
\item Subseccion dedicada unicamente a la concurrencia y comunicacion entre hilos.
\item subseccion para la arquitectura?
\end {itemize}


%-------------------------------------------------------------------
\section{Unity}
%-------------------------------------------------------------------

\begin {itemize}
\item Uso de Unity (entidades, ciclo de vida de entidades, lenguaje de scripting usando .NET)
\item Usos habituales del QR aqui? Mejor hacerlo en el apartado de Android?
\item Uso de las hebras para el tratamiento de los datos que se leen desde la hebra de red que tenemos pendiente de la llegada de datos al socket.
\item Uso de una hebra para mandar datos por la red.
\item Subseccion para explicar la utilizacion de esto?
\item subseccion para la arquitectura?
\end {itemize}


% Variable local para emacs, para  que encuentre el fichero maestro de
% compilaci�n y funcionen mejor algunas teclas r�pidas de AucTeX
%%%
%%% Local Variables:
%%% mode: latex
%%% TeX-master: "../ManualTeXiS.tex"
%%% End:
