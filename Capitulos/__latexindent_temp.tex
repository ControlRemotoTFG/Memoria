%---------------------------------------------------------------------
%
%                          Cap�tulo 1
%
%---------------------------------------------------------------------
%
% 01Introduccion.tex
% Copyright 2009 Marco Antonio Gomez-Martin, Pedro Pablo Gomez-Martin
%
% This file belongs to the TeXiS manual, a LaTeX template for writting
% Thesis and other documents. The complete last TeXiS package can
% be obtained from http://gaia.fdi.ucm.es/projects/texis/
%
% Although the TeXiS template itself is distributed under the 
% conditions of the LaTeX Project Public License
% (http://www.latex-project.org/lppl.txt), the manual content
% uses the CC-BY-SA license that stays that you are free:
%
%    - to share & to copy, distribute and transmit the work
%    - to remix and to adapt the work
%
% under the following conditions:
%
%    - Attribution: you must attribute the work in the manner
%      specified by the author or licensor (but not in any way that
%      suggests that they endorse you or your use of the work).
%    - Share Alike: if you alter, transform, or build upon this
%      work, you may distribute the resulting work only under the
%      same, similar or a compatible license.
%
% The complete license is available in
% http://creativecommons.org/licenses/by-sa/3.0/legalcode
%
%---------------------------------------------------------------------

\chapter{Introducci\'on}

\begin{FraseCelebre}
\begin{Frase}
%P�sose don Quijote delante de dicho carro, y haciendo en su fantas�a
%uno de los m�s desvariados discursos que jam�s hab�a hecho, dijo en
%alta voz:
\end{Frase}
\begin{Fuente}
 % Alonso Fern�ndez de Avellaneda, El Ingenioso Hidalgo Don Quijote de
  %la Mancha
\end{Fuente}
\end{FraseCelebre}

%-------------------------------------------------------------------
\section{Introducci\'on}
%-------------------------------------------------------------------
\label{cap1:sec:introduccion}

La industria del videojuego desde su inicio nos ha ense'nado que la innovaci\'on
 a la hora de crear experiencias nuevas para los usuarios es algo que enriquece a muchos jugadores,por ende se est\'a ayudando a que
la experiencia de juego sea cada vez m\'as c\'omoda y flexible ppara los jugadores.

Es por esta raz\'on que empresas como Electronic Arts, Ubisoft, Kunos Simulazioni y Polyphony Digital,entre otras, dedican gran cantidad de sus recursos
a hacer realidad muchas experiencias que los usuarios quieren tener como, por ejemplo,
 jugar a juegos deportivos realistas como en FIFA o conducir automoviles de competici\'on por un circuito famoso como en Assetto Corsa.

Pero para invertir en crear este nuevo tipo de estilos de juego se necesita una gran financiaci\'on,
 cosa que estudios peque'nos no tienen.  Estos estudios independientes usan motores como Unity3D o Unreal Engine 4.
 Estos motores ,cuyo pago es porcentual a las ganancias obtenidas por tu juego o en algunos casos mensual, ofrecen una gran cantidad de herramientas a disposici\'on 
de sus usuarios para que los desarrolladores puedan ahorrar tiempo de implementaci\'on de una nueva caracter\'isticas y lo utilic\'en para que su juego siga creciendo.
En el caso de Unity el pago es mensual dependiendo de la cantidad de dinero generado por el usuario o su empresa;
 los precios van desde gratuito si hace menos de 100 mil d\'olares anuales hasta 125 d\'olares mensuales si el usuario o su empresa genera m\'as de 200 mil d\'olares
  anuales. 
  Si hablamos del pago de Unreal Engine, estos reciben el 5\% de todo los dinero generado por cada uno de los videojuegos creados con dicho motor cada vez que estos pasen de los 3 mil d\'olares generados en cada cuarto de a'no.  
El desarrollo para dispositivos m\'oviles en estudios independientes ha crecido de manera exponencial gracias a que motores como Unity lo hacen bastante accesible. 


Entonces,
 ?`Y si un juego se pudiese jugar en el tel\'efono m\'ovil pero en verdad el juego se estuviese ejecutando en el ordenador?


%-------------------------------------------------------------------
\section{Proposici\'on de la idea}
%-------------------------------------------------------------------
\label{cap1:sec:porp-idea}

Este trabajo pretende usar un m\'ovil como mando inal\'ambrico usando como motor Unity. Se busca crear un h\'ibrido entre un juego convencional de ordenador
 junto con una tecnolog\'ia, la cual la mayor parte de los usuarios tiene, como son los tel\'efonos m\'oviles.

La herramienta propuesta se apoyar\'a sobre Unity, motor usado en el 34 por ciento del top mil de juegos de moviles y en el 53 por cierto de juegos de Oculus Rift en su salida,
 sirviendo como una prueba de concepto para integrar el uso de un dispositivo m\'ovil a un juego convencional de ordenador.

Una vez terminada,  la herramienta podr\'a ser usada por cualquier desarrollador que lo desee. La herramienta ser\'a de licencia libre para su posible ampliaci\'on. 

%-------------------------------------------------------------------
\section{Estructura de cap\'itulos}
%-------------------------------------------------------------------
\label{cap1:sec:estructura}

Este documento est\'a estructurado en los siguientes cap\'itulos:

\begin{itemize}
\item El cap\'itulo~\ref{cap2} expone una revisi\'on del estado del arte de las tecnolog\'ias involucradas en el desarrollo de este trabajo.

\item El cap\'itulo~\ref{cap3} se centra en los objetivos a cumplir en este proyecto junto con la especificaci\'on y las metodolog\'ias que se van a seguir para la completa realizaci\'on de este trabajo.

\item El cap\'itulo~\ref{cap4} explica todo lo relacionado con el dise'no de la herramienta y la posterior implementaci\'on de la misma. Tambi\'en incluye la explicaci\'on de los prototipos llevados a cabo durante el desarrollo del proyecto.

\item El cap\'itulo~\ref{cap5} aborda los resultados obtenidos y la demostraci\'on de lo conseguido.

\item El cap\'itulo~\ref{cap6} expone la discus\'on y las conslusiones.
\end{itemize}

%El manual tiene, por �ltimo, un ap�ndice que, si bien no es
%interesante desde el punto de vista del usuario, nos sirve de excusa
%para proporcionar el c�digo \LaTeX\ necesario para su creaci�n: a modo
%de ``as� se hizo'', comenta brevemente c�mo fue el proceso de
%escritura de nuestras tesis.



% Variable local para emacs, para  que encuentre el fichero maestro de
% compilaci�n y funcionen mejor algunas teclas r�pidas de AucTeX
%%%
%%% Local Variables:
%%% mode: latex
%%% TeX-master: "../ManualTeXiS.tex"
%%% End:
