%---------------------------------------------------------------------
%
%                          Cap�tulo 1
%
%---------------------------------------------------------------------
%
% 01Introduccion.tex
% Copyright 2009 Marco Antonio Gomez-Martin, Pedro Pablo Gomez-Martin
%
% This file belongs to the TeXiS manual, a LaTeX template for writting
% Thesis and other documents. The complete last TeXiS package can
% be obtained from http://gaia.fdi.ucm.es/projects/texis/
%
% Although the TeXiS template itself is distributed under the 
% conditions of the LaTeX Project Public License
% (http://www.latex-project.org/lppl.txt), the manual content
% uses the CC-BY-SA license that stays that you are free:
%
%    - to share & to copy, distribute and transmit the work
%    - to remix and to adapt the work
%
% under the following conditions:
%
%    - Attribution: you must attribute the work in the manner
%      specified by the author or licensor (but not in any way that
%      suggests that they endorse you or your use of the work).
%    - Share Alike: if you alter, transform, or build upon this
%      work, you may distribute the resulting work only under the
%      same, similar or a compatible license.
%
% The complete license is available in
% http://creativecommons.org/licenses/by-sa/3.0/legalcode
%
%---------------------------------------------------------------------

\chapter{Introducci\'on}

Los motores de videojuegos son herramientas utilizadas por empresas de todo tipo, no solamente de videojuegos. Cortos de animaci\'on, anuncios o incluso \textit{overlays} utilizados en programas televisivos son algunas de las aplicaciones que pueden hacerse con motores como Unity o Unreal. El uso de estos motores de videojuegos para la realizaci\'on de otro tipo de proyectos es algo que se expande cada vez m\'as y con la generaci\'on de aplicaciones para dispositivos m\'oviles todo ha crecido de manera exponencial.

%-------------------------------------------------------------------
\section{Motivaci\'on}
%-------------------------------------------------------------------
\label{cap1:sec:introduccion}

La industria del videojuego desde su inicio nos ha ense'nado que la innovaci\'on
 a la hora de crear experiencias nuevas para los usuarios es algo que enriquece a muchos jugadores,por ende se est\'a ayudando a que
la experiencia de juego sea cada vez m\'as c\'omoda y flexible para los jugadores.

Es por esta raz\'on que empresas como Electronic Arts, Ubisoft, Kunos Simulazioni y Polyphony Digital,entre otras, dedican gran cantidad de sus recursos
a hacer realidad muchas experiencias que los usuarios quieren tener como, por ejemplo,
 jugar a juegos deportivos realistas como en FIFA o conducir automoviles de competici\'on por un circuito famoso como en Assetto Corsa.

Pero para invertir en crear este nuevo tipo de estilos de juego se necesita una gran financiaci\'on,
 cosa que estudios peque'nos no tienen.  Estos estudios independientes usan motores como Unity3D o Unreal Engine 4.
 Estos motores, cuyo pago es porcentual a las ganancias obtenidas por tu juego o en algunos casos mensual, ofrecen una gran cantidad de herramientas a disposici\'on 
de sus usuarios para que los desarrolladores puedan ahorrar tiempo de implementaci\'on de una nueva caracter\'isticas y lo utilic\'en para que su juego siga creciendo.
En el caso de Unity el pago es mensual dependiendo de la cantidad de dinero generado por el usuario o su empresa;
 los precios van desde gratuito si hace menos de 100 mil d\'olares anuales hasta 125 d\'olares mensuales si el usuario o su empresa genera m\'as de 200 mil d\'olares
  anuales. 
  Si hablamos del pago de Unreal Engine, si cualquiera de los productos realizados por el estudio se comercializa de forma oficial, Epic Games obtendr\'ia el 5 \% de los beneficios de la obra cada trimestre cuando este producto supere sus primeros 3000 d\'olares.
El desarrollo para dispositivos m\'oviles en estudios independientes ha crecido de manera exponencial gracias a que motores como Unity lo hacen bastante accesible. 

Unity es utilziado por el 34\% del top mil de juegos para dispositivos m\'oviles. Algunos de estos juegos son \textit{Alto's Adventure}, \textit{Monument Valley} o el famoso \textit{Hearthstone} de Blizzard en su versi\'on de Android e iOS.

Entonces,
 ?`y si un juego se pudiese jugar en el tel\'efono m\'ovil pero en verdad el juego se estuviese ejecutando en el ordenador?

Con esta pregunta se pretende poner encima de la mesa las tecnolog\'ias desarrolladas por diferentes empresas como \textbf{Nintendo} con Nintendo Switch que alterna modo port\'atil con modo sobremesa, lo que da una flexibilidad a la hora de jugar donde quieras que nunca se hab\'ia experimentado, 
como \textbf{Sony} con sus aplicaciones \textbf{PS4 Remote Play} y \textbf{PS4 Second Screen} que te dan la posibilidad de controlar de manera remota e incluso jugar desde tu dispositivo iOS o Android.

Para la conexi\'on de una consola o un ordenador a un dispositivo m\'ovil o tablet es necesaria una conexi\'on a internet estable. El tiempo de respuesta, lo que se conoce como  \textit{Input Lag}, puede llegar a convertirse en un quebradero de cabeza para un estudio peque\~no ya que requiere de pruebas de rendimiento, pruebas con usuarios y pruebas con diferentes anchos de banda de red para buscar posibles cuellos de botella. 

Este trabajo pretende aplicar los modelos propuestos anteriormente y crear una herramienta libre para el motor de videojuegos Unity con la finalidad de dar a estudios independientes y con escasa o nula financiaci\'on la oportunidad de habilitar nuevo gameplay para sus usarios. La finalidad es dar todas estas herramientas de una forma r\'apida e intuitiva con documentaci\'on y una prueba de implementaci\'on de este Plugin con uno de los juegos que se ofrecen de manera gratuita en la tienda de Unity, la Asset Store. Adem\'as de esto, la herramienta consta de una licencia libre, lo que da la posibilidad de ampliaci\'on y modificaci\'on dependiendo de las necesidades de cada usuario/estudio.

\section{Objetivos}




\section{Estructura del documento}
%El manual tiene, por �ltimo, un ap�ndice que, si bien no es
%interesante desde el punto de vista del usuario, nos sirve de excusa
%para proporcionar el c�digo \LaTeX\ necesario para su creaci�n: a modo
%de ``as� se hizo'', comenta brevemente c�mo fue el proceso de
%escritura de nuestras tesis.



% Variable local para emacs, para  que encuentre el fichero maestro de
% compilaci�n y funcionen mejor algunas teclas r�pidas de AucTeX
%%%
%%% Local Variables:
%%% mode: latex
%%% TeX-master: "../ManualTeXiS.tex"
%%% End:
