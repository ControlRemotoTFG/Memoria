%---------------------------------------------------------------------
%
%                          Cap�tulo 1
%
%---------------------------------------------------------------------
%Cambios en esta versión:
% Primer párrafo de introducción.
%---------------------------------------------------------------------




\chapter{Introducci\'on}

La tecnolog\'ia se ha vuelto indispensable en pr\'acticamente todos los \'ambitos de nuestra sociedad. Son pocos los escenarios  en los que no se vea involucrado un aparato tecnol\'ogico. Tareas tan cotidianas como preparar un caf\'e por la ma\~nana o levantar una persiana ahora son posibles con una aplicaci\'on en nuestro m\'ovil o con un comando por voz. \\

Adem\'as de para facilitar nuestras tareas, los dispositivos m\'oviles se han unido a los ordendores y las consolas de videojuegos en la tarea de entretener a una gran cantidad de usuarios. Con la progresi\'on en la calidad de los dispositivos m\'oviles, la industria del videojuego ha decidido unirse y comenzar a lanzar juegos con grandes presupuestos al mercado m\'ovil. \\

Como veremos, la entrada de los dispositivos m\'oviles en la industria del entretenimiento no ha sido solo en forma de plataforma de juego sino como un aliado de los dispositivos ya existentes. Los dispositivos m\'oviles han ayudado a la industria del entretenimiento a ampliar el abanico de opciones que los usuarios tienen para jugar e interactuar con las consolas actuales. \\


Con este proyecto se pretende explorar la situaci\'on actual del uso de dispositivos m\'oviles como dispositivo de entrada para videojuegos ejecutados en otra plataforma y las motivaciones que existen para invertir e investigar en este nuevo modelo de interacci\'on de usuario. Junto con esta investigaci\'on, se llevar\'a a cabo con un proceso de desarrollo de una herramienta que haga posible jugar utilizando un dispositivo m\'ovil como dispositivo de entrada para videojuegos.


%-------------------------------------------------------------------
\section{Motivaci\'on}
%-------------------------------------------------------------------

Uno de los pilares que siempre ha caracterizado a la industria del videojuego es la innovaci\'on a la hora de crear nuevas experiencias para los usuarios. Son experiencias que enriquecen a muchos jugadores y que cada vez se disfrutan de una manera m\'as c\'omoda y flexible. Uno de los culpables del aumento de esta flexibilidad a la hora de jugar son los dispositivos m\'oviles. \\

En esta \'ultima d\'ecada el mercado del \textit{gaming} ha acogido al tel\'efono m\'ovil como su nuevo integrante. Los juegos para m\'oviles han tenido mucho \'exito entre las nuevas generaciones de jugadores. Tal y como se muestra en el informe publicado por \citep{AEVI2019}, el 45\% de los ingresos de la industria del videojuegos en el a\~no 2019 provino de los videojuegos m\'oviles donde se incluyen tanto dispositivos m\'oviles como tablets. \citep{moviles} plantea si estos nuevos dispositivos m\'oviles son una amenza para las consolas actuales y presenta el caso de varios fabricantes de perif\'ericos \textit{gaming} que se han lanzado al mundo de la fabricaci\'on de \textbf{smartphones} orientados a jugar. \\

T\'itulos relevantes dentro de la industria como \textit{League of Legends}, \textit{Call of Duty} o \textit{Fortnite} ya tienen su versi\'on para dispositivos m\'oviles. Gracias a que estos t\'itulos son gratuitos y se encuentran a la cabeza de los \textit{e-sports} en sus modalidades de PC y consola \citep*{TEOQ32020}, su relevancia dentro de las plataformas m\'oviles ha sido aun mayor. A pesar de esto, tal y como cuenta \citep{futuro} en un art\'iculo, existen varios limitantes en el \textit{gaming} para dispositivos m\'oviles. Algunos de estos limitantes son la bateria de los dispositivos y la necesidad de una conexi\'on estable a internet.\\

Para resolver estos inconvenientes, las principales desarrolladoras de dispositivos m\'oviles comenzaron a fabricar las gamas altas de estos dispositivos que supl\'ian los problemas de bateria, calentamiento y frecuencia de refresco de las pantallas. Sony en particular ha apostado por el desarrollo de una serie de dispositivos m\'oviles pensados para largas sesiones de juego\footnote{Xperia -  \url{https://www.sony.es/electronics/xperia-mobile-gaming}}. Adem\'as de esto, Sony ha desarrollado su aplicaci\'on para poder jugar de manera remota a PlayStation~4 y PlayStation~5, \textbf{PS Remote Play\footnote{\url{https://remoteplay.dl.playstation.net/remoteplay/lang/es/index.html}}}. Adem\'as de para poder jugar, Sony ha desarrollado otra aplicaci\'on que permite a los usuarios controlar la interfaz de la consola PlayStation~4 desde su dispositivo m\'ovil haciendo que este simule ser un mando y una segunda pantalla, \textbf{PS4 Second Screen\footnote{\url{https://play.google.com/store/apps/details?id=com.playstation.mobile2ndscreen&hl=es&gl=US}}}.
Nintendo por su parte ha desarrollado la aplicaci\'on \textbf{Nintendo Switch Online}\footnote{\url{https://www.nintendo.es/Familia-Nintendo-Switch/Nintendo-Switch-Online/Aplicacion-para-moviles-1374628.html}} para llevar a cabo la comunicaci\'on en sus juegos online. Esta aplicaci\'on convierte tu dispositivo m\'ovil en un chat de voz y texto, lo que suple la falta de micr\'ofono incorporado en la consola y permite a los jugadores comunicarse con sus compa\~neros en los juegos online. Microsoft ha desarrollado su aplicaci\'on \textbf{Xbox}\footnote{\url{https://www.xbox.com/es-ES/consoles/remote-play}} con la que poder jugar de manera remota a su consola. Gracias a esta aplicaci\'on, el usuario puede descargarse los juegos en su Xbox, ejecutarlos y conectar su m\'ovil o tablet para poder jugar directamente en su smartphone a trav\'es de internet.\\

Despu\'es de desarrollar sus aplicaciones, Sony sac\'o al mercado una serie de juegos conocidos como \textbf{PlayLink\footnote{\url{https://www.playstation.com/es-es/accessories/playlink/}}}. Este nuevo tipo de juegos se concentran en una colecci\'on de t\'itulos que tienen como caracter\'istica com\'un que no es necesario usar un mando convencional de PlayStation. Estos t\'itulos se juegan directamente usando el tel\'efono m\'ovil como mando y el \'unico requisito es tener cada uno de los dispositivos m\'oviles conectados a la consola via Wi-Fi. Esto soluciona por completo el problema de la escasez de mandos de consola en los hogares ya que \'unicamente ser\'an necesarios los tel\'efonos m\'oviles de las personas que vayan a jugar.\\

El prop\'osito de este trabajo consiste en lograr utilizar un m\'ovil como dispositivo de entrada en un juego de PC, imitando las caracter\'isticas de algunas de las aplicaciones anteriormente mencionadas. Para poder lograrlo, se propone crear una librer\'ia de uso libre para el motor de videojuegos Unity. Esto permitir\'a la posibilidad de ampliaci\'on y modificaci\'on de la librer\'ia dependiendo de las necesidades de cada desarrollo.


\section{Objetivos}

La finalidad del proyecto es la conexi\'on entre 2 dispositivos, uno de ellos que ejecuta el juego y el otro funciona como un dispositivo de entrada / mando para controlar el videojuego. Esta conexi\'on debe ser estable, con el m\'inimo retraso posible y que sea f\'acil de incorporar en proyectos ya desarrollados.\\

El dispositivo m\'ovil tiene la funci\'on de ser el dispositivo de entrada del videojuego. Para lograr esto, en la pantalla del m\'ovil se muestra un mando virtual. Una vez el usuario interactue con este mando virtual, las pulsaciones de los botones que se encuentran en la pantalla se envian al juego a modo de entrada de usuario. Adem\'as de esto, el mando virtual que se muestra puede configurarse desde el juego envi\'andole al dispositivo m\'ovil la im\'agen del mando que mostrar.\\

\defcitealias{libroblanco2019}{Libro blanco del desarrollo espa\~nol de videojuegos 2019}
Con respecto a las plataformas donde realizar ambas aplicaciones, hemos decidido utilizar Android de manera nativa para el dispositivo de entrada y Unity para la parte del juego. Se ha decidido hacer el plug-in para Unity ya que es uno de los motores de juegos m\'as utilizado actualmente. En Espa\~na el 83\% de las empresas de videojuegos utilizan este motor para desarrollar sus juegos \citepalias{libroblanco2019}. El uso de Android para la aplicaci\'on de m\'ovil se debe a las facilidades que ofrece Google para subir aplicaciones a la plataforma Play Store y al amplio parque de dispositivos Android que existen actualmente.\\

Para conseguir este objetivo se ha divido el objetivo final en los siguientes pasos:

\begin {itemize}
\item Desarrollar y publicar un \textit{plug-in open source} para Unity que permita establecer conexi\'on y recibir \textit{input} desde otro dispositivo.
\item Desarrollar y publicar una aplicaci\'on \textit{open source} para Android que permita establecer conexi\'on con otro dispositivo para ser usado como dispositivo de entrada.
\item Evidenciar y realizar un estudio posterior de los resultados del proyecto a trav\'es de una serie de pruebas con usuarios. 
\end {itemize}

Para el \'ultimo punto, se va a realizar un juego simple que pueda ser controlado con un dispositivo m\'ovil para probar que la librer\'ia desarrollada funciona. Una vez el juego est\'e terminado, se va a realizar una posterior prueba con usuarios para obtener una serie de datos como latencia de red, uso del procesador y tiempos de compresi\'on y descompresi\'on de im\'agenes. Los datos recolectados nos permitir\'an averiguar si el sistema desarrollado es apto para uso en juegos comerciales y futuros desarrollos.

\section{Metodolog\'ia}

Como metodolog\'ia de desarrollo se ha decidido usar una metodolog\'ia \'agil de producci\'on que es habitual en la industria del desarrollo de software y videojuegos. Se ha elegido una metodolog\'ia \'agil por la experiencia positiva en proyectos previos.\\

\textbf{Scrum}, propuesto por \citep{scrum} es un framework para la gesti\'on de proyectos de trabajo en equipo. El texto original proviene de un congreso (OOPSLA 1995) pero en \'el no se incluyen muchos de los t\'erminos que hoy se asocian con la metodolog\'ia SCRUM como las reuniones diarias \citep{scrum2}. Scrum introduce el t\'ermino \textit{sprint} para referirse a periodos de tiempo de entre 2 y 4 semanas de duraci\'on en la que el equipo de desarrollo se compromete a realizar una serie de tareas. Estas tareas no deben ser cambiadas durante la duraci\'on del sprint y el progreso de estas tareas debe ser compartido en las diferentes reuniones de scrum diarias. Una vez finalizado el sprint, debe de planificarse el siguiente con el objetivo de mejorar el producto ya existente. Clinton Keith explica c\'omo introducir esta metodolog\'ia en cada uno de los equipos de desarrollo que configuran un estudio de videojuegos. Actualmente los videojuegos son productos que tardan varios a\~nos en desarrollarse e involucran grandes presupuestos, es por esto que revisar el producto sobre el que se est\'a trabajando cada poco tiempo ayuda a las diferentes partes del equipo a tener una visi\'on m\'as global del videojuego \citep{keith2010agile}.  \\

Debido a los problemas de disponibilidad durante el desarrollo del proyecto, se realizaba una peque\~na reuni\'on de 10 minutos cada 1-3 d\'ias para ver el progreso de cada uno de los integrantes. En estas reuniones se revisaba la planificaci\'on para la siguiente reuni\'on o la siguiente semana. Al principio del desarrollo fueron necesarias reuniones mucho m\'as largas que en muchas ocasiones incluian a los directores del TFG en las que se discut\'ian las diferentes caracter\'isticas que deber\'ian tener las aplicaciones que se estaban desarrollando. Estas reuniones m\'as extensas serv\'ian como cierre de \textit{Sprint} y como preparaci\'on del siguiente. El seguimiento de las diferentes tareas a realizar se realizaba usando la herramienta online \textbf{Pivotal Tracker} donde se marcaban las tareas con 3 posibles estados: ``Open'', ``In Progress'' o ``Done''.


\section{Planificaci\'on}

La planificaci\'on del desarrollo de este proyecto se ha dividido en 3 fases:\\


\textbf{Documentaci\'on y dise\~no:} Durante esta fase tratamos de delimitar claramente el alcance y objetivos del TFG, reunir fuentes y referencias y preparar las herramientas que se utilizar\'ian durante el resto del desarrollo. Adem\'as, en esta primera fase se realiz\'o un dise\~no de las aplicaciones que se desarrollar\'ian en los meses posteriores.\\

\textbf{Desarrollo:} Durante esta fase del desarrollo se realizaron todas las funcionalidades especificadas en la fase anterior del proyecto. Al hacerse revisiones peri\'odicas de la implementaci\'on, algunas de las funcionalidades iniciales sufrieron cambios o fueron eliminidas y se a\~nadieron otras que encajaban m\'as con el rumbo que estaba tomando el desarrollo. 
Esta fase ha ocupado la mayor parte del tiempo que ha tomado realizar este proyecto. Durante esta fase se realizaron 2 aplicaciones en forma de demo con las que poder probar la herramienta y demostrar la viabilidad del proyecto.\\

\textbf{Cierre:} Durante la fase final del desarrollo se realizaron las mejoras finales de las aplicaciones y se refinaron los \'ultimos detalles de rendimiento. En esta fase tambi\'en se realizaron las pruebas con usuarios para extraer datos tanto de rendimiento de las aplicaciones como de posibles fallos y mejoras de las aplicaciones. Estos datos se han refinado, filtrado y analizado y han servido para extraer las conclusiones finales de este trabajo. Adem\'as, durante esta \'ultima fase del desarrollo se ha trabajado en terminar la redacci\'on de este documento junto con la revisi\'on por los directores de este TFG. 

\section{Estructura del documento}

Este proyecto se divide en 6 cap\'itulos, cada uno de ellos dedicado a una tem\'atica. Esta secci\'on est\'a situada en el cap\'itulo de introducci\'on donde se ha definido la motivaci\'on y los objetivos del proyecto.\\

El cap\'itulo 2 recoge el estudio inicial del estado del arte en el cual se exponen los antecedentes de la tem\'atica del proyecto. En este cap\'itulo se mencionan y explican los cambios que han sufrido los diferentes dispositivos de entrada para videojuegos desde los inicios. En este cap\'itulo se incluye tambi\'en el funcionamiento de los sistemas de streaming y la retroalimentaci\'on en los controladores de videojuegos.\\

En el cap\'itulo 3 se explica todo lo relacionado con la especificaci\'on de las aplicaciones que se van a desarrollar. Esta especificaci\'on incluye una descripci\'on del protocolo de comunicaci\'on entre dispositivos necesario para este proyecto.\\

En el cap\'itulo 4 se explica de manera detallada la implementaci\'on de las aplicaciones que se han desarrollado para este proyecto.\\

El cap\'itulo 5 se explica la demo desarrollada para probar la viabilidad del \textit{plug-in} desarrollado y como caso pr\'actico para la presentaci\'on de este proyecto. Adem\'as, se recopila el peque\~no experimento que se ha llevado a cabo con diferentes usuarios para probar la aplicaci\'on y recopilar \textit{feedback} de los diferentes usuarios que han probado la demo. Tambi\'en se describen los participantes, los resultados obtenidos y la discusi\'on sobre estos resultados.\\

En el \'ultimo cap\'itulo se explican de manera detalla las conclusiones obtenidas despu\'es de realizar el proyecto y una visi\'on general del trabajo futuro que inspira este proyecto.



%El manual tiene, por �ltimo, un ap�ndice que, si bien no es
%interesante desde el punto de vista del usuario, nos sirve de excusa
%para proporcionar el c�digo \LaTeX\ necesario para su creaci�n: a modo
%de ``as� se hizo'', comenta brevemente c�mo fue el proceso de
%escritura de nuestras tesis.



% Variable local para emacs, para  que encuentre el fichero maestro de
% compilaci�n y funcionen mejor algunas teclas r�pidas de AucTeX
%%%
%%% Local Variables:
%%% mode: latex
%%% TeX-master: "../ManualTeXiS.tex"
%%% End:
