%---------------------------------------------------------------------
%
%                          Cap�tulo 1
%
%---------------------------------------------------------------------
%Cambios en esta versión:
% Primer párrafo de introducción.
%---------------------------------------------------------------------

\chapter{Introducci\'on}

La relaci\'on entre los humanos y las m\'aquinas se ha ido consolidando en las \'ultimas d\'ecadas hasta el punto que son imprescindibles para algunas de nuestras tareas cotidianas. La tecnolog\'ia extiende nuestras capacidades y durante las \'ultimas d\'ecadas se han ido forjando y mejorando las interfaces de comunicaci\'on hombre-m\'aquina. Las pantallas t\'actiles y los controles por voz son algo a lo que la sociedad se ha acostumbrado y una gran parte de la responsabilidad de este hecho reside en los dispositivos m\'oviles. Junto con esta constante mejora en las interfaces humano-m\'aquina tambi\'en puede encontrarse una introducci\'on a la tecnolog\'ia h\'aptica en modo de pulso para saber que el tel\'efono ha registrado una orden o pulsaci\'on en la pantalla.

%-------------------------------------------------------------------
\section{Motivaci\'on}
%-------------------------------------------------------------------
\label{cap1:sec:introduccion}

La industria del videojuego desde su inicio nos ha ense'nado que la innovaci\'on
 a la hora de crear experiencias nuevas para los usuarios es algo que enriquece a muchos jugadores, por ende se est\'a ayudando a que
la experiencia de juego sea cada vez m\'as c\'omoda y flexible para los jugadores.

Es por esta raz\'on que empresas como Electronic Arts, Ubisoft, Kunos Simulazioni y Polyphony Digital,entre otras, dedican gran cantidad de sus recursos
a hacer realidad muchas experiencias que los usuarios quieren tener como, por ejemplo,
 jugar a juegos deportivos realistas como en FIFA o conducir automoviles de competici\'on por un circuito famoso como en Assetto Corsa.

Pero para invertir en crear este nuevo tipo de estilos de juego se necesita una gran financiaci\'on,
 cosa que estudios peque'nos no tienen.  Estos estudios independientes usan motores como Unity3D o Unreal Engine 4.
 Estos motores, cuyo pago es porcentual a las ganancias obtenidas por tu juego o en algunos casos mensual, ofrecen una gran cantidad de herramientas a disposici\'on 
de sus usuarios para que los desarrolladores puedan ahorrar tiempo de implementaci\'on de una nueva caracter\'isticas y lo utilic\'en para que su juego siga creciendo.
En el caso de Unity el pago es mensual dependiendo de la cantidad de dinero generado por el usuario o su empresa;
 los precios van desde gratuito si hace menos de 100 mil d\'olares anuales hasta 125 d\'olares mensuales si el usuario o su empresa genera m\'as de 200 mil d\'olares
  anuales. 
  Si hablamos del pago de Unreal Engine, si cualquiera de los productos realizados por el estudio se comercializa de forma oficial, Epic Games obtendr\'ia el 5 \% de los beneficios de la obra cada trimestre cuando este producto supere sus primeros 3000 d\'olares.
El desarrollo para dispositivos m\'oviles en estudios independientes ha crecido de manera exponencial gracias a que motores como Unity lo hacen bastante accesible. 

Unity es utilziado por el 34\% del top mil de juegos para dispositivos m\'oviles. Algunos de estos juegos son \textit{Alto's Adventure}, \textit{Monument Valley} o el famoso \textit{Hearthstone} de Blizzard en su versi\'on de Android e iOS.

Entonces,
 ?`y si un juego se pudiese jugar en el tel\'efono m\'ovil pero en verdad el juego se estuviese ejecutando en el ordenador?

Con esta pregunta se pretende poner encima de la mesa las tecnolog\'ias desarrolladas por diferentes empresas como \textbf{Nintendo} con Nintendo Switch que alterna modo port\'atil con modo sobremesa, lo que da una flexibilidad a la hora de jugar donde quieras que nunca se hab\'ia experimentado, 
como \textbf{Sony} con sus aplicaciones \textbf{PS4 Remote Play} y \textbf{PS4 Second Screen} que te dan la posibilidad de controlar de manera remota e incluso jugar desde tu dispositivo iOS o Android.

Para la conexi\'on de una consola o un ordenador a un dispositivo m\'ovil o tablet es necesaria una conexi\'on a internet estable. El tiempo de respuesta, lo que se conoce como  \textit{Input Lag}, puede llegar a convertirse en un quebradero de cabeza para un estudio peque\~no ya que requiere de pruebas de rendimiento, pruebas con usuarios y pruebas con diferentes anchos de banda de red para buscar posibles cuellos de botella. 

Este trabajo pretende aplicar los modelos propuestos anteriormente y crear una herramienta libre para el motor de videojuegos Unity con la finalidad de dar a estudios independientes y con escasa o nula financiaci\'on la oportunidad de habilitar nuevo gameplay para sus usarios. La finalidad es dar todas estas herramientas de una forma r\'apida e intuitiva con documentaci\'on y una prueba de implementaci\'on de este Plugin con uno de los juegos que se ofrecen de manera gratuita en la tienda de Unity, la Asset Store. Adem\'as de esto, la herramienta consta de una licencia libre, lo que da la posibilidad de ampliaci\'on y modificaci\'on dependiendo de las necesidades de cada usuario/estudio.

\section{Objetivos}

Los objetivos de este proyecto son:

\begin {itemize}
\item Desarrollar y publicar un \textit{plug-in open source} para Unity que permita establecer conexi\'on y recibir \textit{input} desde otro dispositivo.
\item Desarrollar y publicar una aplicaci\'on \textit{open source} para Android que permita establecer conexi\'on con otro dispositivo para ser usado como dispositivo de entrada.
\item Evidenciar y realizar un estudio posterior de los resultados del proyecto atrav\'es de una serie de pruebas con usuarios. 
\end {itemize}

Dentro de Unity existe una plataforma  en la que los diferentes creadores de contenido suben sus creaciones para que otros desarrolladores las utilicen para sus proyectos. Estas herramientas pueden ser tanto de pago como gratuitas y entre ellas se encuentran modelos 3D, recursos de audio como m\'usica y efectos de sonido y \textit{plug-ins}. Un \textit{plug-in} es \textit{software} que se introduce dentro de un programa para a\~nadir nuevas funcionalidades. 

\section{Metodolog\'ia}

Como metodolog\'ia de desarrollo se ha decidido usar una metodolog\'ia \'agil de producci\'on que es habitual en la industria del desarrollo de software y videojuegos. Se ha elegido una metodolog\'ia \'agil por la experiencia positiva en proyectos previos.

\textbf{Scrum}, propuesto por Schwaber and Sutherland (2018), es un framework para la gesti\'on de proyectos de trabajo en equipo. El paradigma se basa en un principio simple: comenzar con metas a corto plazo que formen parte del resultado final, tras esto se sigue el progreso y modifica seg\'un se avance en el proyecto.

Debido a los problemas de disponibilidad durante el desarrollo del proyecto, se realizaba una peque\~na reuni\'on de 10 minutos cada 1-3 d\'ias para ver el progreso de cada uno de los integrantes. En estas reuniones se revisaba la planificaci\'on para la siguiente reuni\'on o la siguiente semana. Al principio del desarrollo fueron necesarias reuniones mucho m\'as largas que en muchas ocasiones incluian a los directores del TFG en las que se discut\'ian las diferentes \textit{features} que deber\'ian tener las aplicaciones que se estaban desarrollando. Estas reuniones m\'as extensas serv\'ian como cierre de \textit{Sprint} y como preparaci\'on del siguiente. El seguimiento de las diferentes tareas a realizar se realizaba usando la herramienta online \textbf{Pivotal Tracker} donde se marcaban las tareas con 3 posibles estados: ``Open'', ``In Progress'' o ``Done''.


\section{Planificaci\'on}

La planificaci\'on del desarrollo de este proyecto se ha dividido en 3 fases:\\


\textbf{Documentaci\'on y dise\~no:} Durante esta fase tratamos de delimitar claramente el alcance y objetivos del TFG, reunir fuentes y referencias y preparar las herramientas que se utilizar\'ian durante el resto del desarrollo. Adem\'as, en esta primera fase se realiz\'o un dise\~no de las aplicaciones que se desarrollar\'ian en los meses posteriores.\\

\textbf{Desarrollo:} Durante esta fase del desarrollo se desarrollaron todas las funcionalidades especificadas en la fase anterior del proyecto. Al hacerse revisiones peri\'odicas de la implementaci\'on, algunas de las funcionalidades iniciales sufrieron cambios o fueron eliminidas y se a\~nadieron otras que encajaban m\'as con el rumbo que estaba tomando el desarrollo. 
Esta fase ha ocupado la mayor parte del tiempo que ha tomado realizar este proyecto. Durante esta fase se realizaron 2 aplicaciones en forma de demo con las que poder probar la herramienta y demostrar la viabilidad del proyecto.\\

\textbf{Cierre:} Durante la fase final del desarrollo se realizaron las mejoras finales de las aplicaciones y se refinaron los \'ultimos detalles de rendimiento. En esta fase tambi\'en se realizaron las pruebas con usuarios para extraer datos tanto de rendimiento de las aplicaciones como de posibles fallos y mejoras de las aplicaciones. Estos datos se han refinado, filtrado y analizado y han servido para extraer las conclusiones finales de este trabajo. Adem\'as, durante esta \'ultima fase del desarrollo se ha trabajado en terminar la redacci\'on de este documento junto con la revisi\'on por los directores de este TFG. 

\section{Estructura del documento}

Este proyecto se divide en 6 cap\'itulos, cada uno de ellos dedicado a una tem\'atica. Esta secci\'on est\'a situada en el cap\'itulo de introducci\'on donde se ha definido la motivaci\'on y los objetivos del proyecto.

El cap\'itulo 2 recoge el estudio inicial del estado del arte en el cual se exponen los antecedentes de la tem\'atica del proyecto. Est\'a dividido en varias secciones, la primera hace un repaso de la evoluci\'on que han sufrido los dispositivos de entrada desde que desarroll\'o el teclado hasta la \'ultima generaci\'on de consolas. La siguiente secci\'on trata la tecnolog\'ia h\'aptica y la retroalimentaci\'on en los controladores de videojuegos. La \'ultima secci\'on expone el funcionamiento de un sistema de streaming en red.

En el cap\'itulo 3 se explica todo lo relacionado con la especificaci\'on de las aplicaciones que se van a desarrollar. Esta especificaci\'on incluye una descripci\'on del protocolo de comunicaci\'on entre dispositivos necesario para este proyecto.

En el cap\'itulo 4 se explica de manera detallada la implementaci\'on de las aplicaciones que se han desarrollado para este proyecto. Junto a la implemetaci\'on, se explica la demo desarrollada para probar la viabilidad del \textit{plug-in} desarrollado y como caso pr\'actico para la presentaci\'on de este proyecto. 

El cap\'itulo 5 recopila el peque\~no experimento que se ha llevado a cabo con diferentes usuarios para probar la aplicaci\'on y recopilar \textit{feedback} de los diferentes usuarios que han probado la demo. Tambi\'en se describen los participantes, los resultados obtenidos y la discusi\'on sobre estos resultados.

En el \'ultimo cap\'itulo se explican de manera detalla las conclusiones obtenidas despu\'es de realizar el proyecto y una visi\'on general del trabajo futuro que inspira este proyecto.


%El manual tiene, por �ltimo, un ap�ndice que, si bien no es
%interesante desde el punto de vista del usuario, nos sirve de excusa
%para proporcionar el c�digo \LaTeX\ necesario para su creaci�n: a modo
%de ``as� se hizo'', comenta brevemente c�mo fue el proceso de
%escritura de nuestras tesis.



% Variable local para emacs, para  que encuentre el fichero maestro de
% compilaci�n y funcionen mejor algunas teclas r�pidas de AucTeX
%%%
%%% Local Variables:
%%% mode: latex
%%% TeX-master: "../ManualTeXiS.tex"
%%% End:
