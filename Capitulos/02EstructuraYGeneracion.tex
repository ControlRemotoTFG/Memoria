%---------------------------------------------------------------------
%
%                          Cap�tulo 2
%
%---------------------------------------------------------------------
%
% 02EstructuraYGeneracion.tex
% Copyright 2009 Marco Antonio Gomez-Martin, Pedro Pablo Gomez-Martin
%
% This file belongs to the TeXiS manual, a LaTeX template for writting
% Thesis and other documents. The complete last TeXiS package can
% be obtained from http://gaia.fdi.ucm.es/projects/texis/
%
% Although the TeXiS template itself is distributed under the 
% conditions of the LaTeX Project Public License
% (http://www.latex-project.org/lppl.txt), the manual content
% uses the CC-BY-SA license that stays that you are free:
%
%    - to share & to copy, distribute and transmit the work
%    - to remix and to adapt the work
%
% under the following conditions:
%
%    - Attribution: you must attribute the work in the manner
%      specified by the author or licensor (but not in any way that
%      suggests that they endorse you or your use of the work).
%    - Share Alike: if you alter, transform, or build upon this
%      work, you may distribute the resulting work only under the
%      same, similar or a compatible license.
%
% The complete license is available in
% http://creativecommons.org/licenses/by-sa/3.0/legalcode
%
%---------------------------------------------------------------------

\chapter{Entrada de usuario en videojuegos}
\label{cap2}

\begin{FraseCelebre}
\begin{Frase}
  %La mejor estructura no garantizar� los resultados ni el rendimiento.
  %Pero la estructura equivocada es una garant�a de fracaso.
\end{Frase}
\begin{Fuente}
%Peter Drucker
\end{Fuente}
\end{FraseCelebre}


%-------------------------------------------------------------------
\section{Evoluci\'on de los dispositivos de entrada}

Los dispositivos de entrada son aquellos que permiten introducir datos o informaci\'on en un ordenador para que este los procese u ordene. Otro t\'ermino usado para estos dispositivos es perif\'erico. A pesar de que este t\'ermino implica a menudo el concepto de adicional y no esencial, en muchos sistemas inform\'aticos son elementos fundamentales. Estos dispositivos de entrada pueden clasificarse seg\'un el tipo de entrada, ya sea sonora, visual, de movimiento mec\'anico, etc. o dependiendo de si su forma de entrada es discreta (pulsaciones de teclas) o continua (una posici\'on).

Todos estos dispositivos de entrada son conocidos como \textbf{HID, Human Interface Device}. La principal motivaci\'on para HID era la de permitir innovaciones en los dispositivos de entrada a los ordenadores y as\'i simplificar el proceso de instalaci\'on de este tipo de dispositivos. Antes de HID, los dispositivos de entrada se ajustaban a unos estrictos protocolos dise\~nados para ratones, teclados y joysticks. Con cualquier innovaci\'on en el hardware se requer\'ia de sobrecargar el protocolo existente o la creaci\'on de un driver. Un solo driver HID analiza los datos de entrada y permite una asociaci\'on din\'amica de estos datos con la funcionalidad descrita por la aplicaci\'on. En el protocolo HID existen 2 enidades: el host y el dispositivo.El dispositivo es la entrada que intercat\'ua directamente con el humano, un ejemplo puede ser el teclado o rat\'on. El host es el que recibe los datos de entrada del dispositivo con las acciones ejecutadas por el humano, el host suele ser un ordenador.
 Los dispositivos definen sus paquetes de datos y luego presentan un descriptor HID al host. El descriptor HID es codificado como un grupo de bytes que describen los paquetes de datos del dispositivo. Esto incluye: cuantos paquetes soporta el dispositivo, el tama\~no de los paquetes, y el prop\'osito de cada byte y bit en el paquete. Se espera que el host sea la entidad m\'as compleja de las 2 ya que el host necesita obtener el descriptor HID del dispositivo y analizarlo antes de establecer la comunicaci\'on. El HID tambi\'en define el modo arranque, este modo es limitado ya que paquetes de datos son utilizados durante ese modo. Los \'unicos dispositivos que soportan el modo arranque son teclados y ratones.\par

Un teclado es un HID que se representa como una disposici\'on de botones o teclas. Cada una de estas teclas se puede utilizar para ingresar cualquier car\'acter ling\"u\'istico o hacer una llamada a cualquier funci\'on particular del ordenador. Los teclados  tienen su inspiraci\'on en las m\'aquinas de escribir que tienen su origen en el a\~no 1877, cuando la marca Remington comenz\'o a comercializar de manera masiva las m\'aquinas de escribir. Uno de los primeros avances de estas m\'aquinas de escribir ocurri\'o en la d\'ecada de 1930, cuando se combinaron la tecnolog\'ia de la entrada e impresi\'on de las m\'aquinas de escribir con la tecnolog\'ia de la comunicaci\'on del tel\'egrafo. Adem\'as, los sistemas de tarjetas perforadas tambi\'en se combinaron con las m\'aquinas de escribir para crear perforadoras de teclado que fueron la base para las primeras calculadoras. La emergente m\'aquina de escribir el\'ectrica mejor\'o a\'un m\'as la uni\'on tecnol\'ogica entre la m\'aquina de escribir y el ordenador. As\'i, en 1955, el Whirlwind del MIT, se convierte en el primer ordenador del mundo que permite a sus usuarios introducir comandos a trav\'es de un teclado y confirma lo \'util y conveniente que puede ser un dispositivo de entrada de teclado. En 1964 se impul\'o el desarrollo de una nueva interfaz de usuario llamada terminal de visualizaci\'on de video que permiti\'o a los usuarios de las primeras computadoras ver qu\'e carcacteres estaban escribiendo. En los a\~nos 80, IBM lanz\'o su primer ordenador personal que inclu\'ia un teclado mec\'anico. Actualmente las principales mejoras que han sufrido los teclados de ordenador se basan en la eliminaci\'on de cables gracias al Bluetooth o WI-FI. Con la llegada de los dispositivos t\'actiles se a\~nadi\'o adem\'as el concepto de teclado virtual. Este teclado virtual elimina el uso de un teclado hardware para pasar a un teclado software que imita el teclado tradicional QWERTY pero en una pantalla t\'actil. \par

Adem\'as del teclado, el segundo dispositivo de entrada por excelencia es el rat\'on. La primera maqueta fue dise\~nada durante los a\~nos 60, dispon\'ia de 2 ruedas met\'alicas que al desplazarse por una superficie mov\'ian 2 ejes. Cada uno de estos ejes controlaba el movimiento tanto vertical como horizontal del cursor en la pantalla. Durante lo s\~nos posteriores se a\~nadieron botones, una rueda central o lateral para el desplazamiento, el sensor de movimiento \'optico por diodo led o un sensor basado en un l\'aser no visible. En los primeros a\~nos de la inform\'atica, el teclado era el dispositivo m\'as popular para la entrada de datos pero la aparici\'on y \'exito del rat\'on, junto con la evoluci\'on de los sistemas operativos, lograron facilitar y mejorar la comodidad a la hora de manejar ambos perif\'ericos a la vez. Los ratones suelen estar preparados para manejarse con ambas manos pero algunos fabricantes tambi\'en ofrecen modelos espec\'icos para zurdos y diestros. Los sistemas operativos tambi\'en han ayudado a que ambos tipos de personas tengan un uso satisfactorio de los ratones. Con el avance de los nuevos ordenadores, el rat\'on se ha convertido en un dipsositivo esencial a la hora de jugar videojuegos, sirviendo no solo para seleccionar y accionar objetos en pantalla en juegos de estrategia, sino para controlar la c\'amara o cambiar la direcci\'on del personaje en juegos de primera y tercera persona.\par

Los videojuegos han sido los principables responsables de la evoluci\'on de los dispositivos de entrada en las d\'ecadas posteriores a los a\~nos 70. En 1972 fue lanzada de manera oficial la \textbf{Magnavox Odyssey} y fue considerada la primera videoconsola. El dispositivo de entrada para poder jugar consit\'ia de 2 diales que se utilizaban para el movimiento horizontal y vertical del personaje, un cuadrado blanco. Como juego para la consola Odyssey sacaron el Magnavox Odyssey Shooting Gallery en 1972. El mando que se usaba para poder jugar a este juego tenía la forma de un rifle y la peculiaridad que tenía este accesorio/mando era que los disparos se registraban siempre y cuando el rifle apuntase a una luz intensa por lo que era muy f\'acil de enga\~nar si no fuese porque el juego no dispon\'ia de un registro de puntos. En 1976 la empresa Fairchild Semiconductor sac\'o al mercado la \textbf{Fairchild Channel F} cuya caracter\'istica principal a nivel de entrada de usuario fue la incorporaci\'on de un joystick de 8 direcciones.  Adem\'as de ofrecer un movimiento en 8 direcciones, la parte de arriba de este mando pod\'ia girarse para ser compatible con juegos como \textit{Pong} y tambi\'en pod\'ia ser pulsado y usarse normalmente como bot\'on de disparo. En 1977 sali\'o al mercado uno de los joysticks m\'as famosos. Este joystick es el que se utilizaba en la consola \textbf{Atari 2600}. Este joystick se conoc\'ia como el \textbf{Atari CX40} y consist\'ia de una palanca que permit\'ia un movimiento en 8 direcciones y un bot\'on. Junto con este modelo, Atari sac\'o al mercado un tipo de conexi\'on que se convertir\'ia en el est\'andar de la ind\'ustria y que ser\'ia compatible con sistemas posteriores. Unos pocos a\~nos despu\'es, en 1982, Atari lanz\'o su nueva consola Atari 5200. Adem\'as de mejorar gr\'aficamente, el controlador que llevaba incorporado proviene de un kit de controlador de un avi\'on RC. El sistema combinaba un dise\~no mec\'anico demasiado complejo con un sistema de circuito flexible interno de muy bajo coste. Este controlador incluy\'o un bot\'on de pausa, una caracter\'istica \'unica en ese momento.\par

En 1983 Nintendo sac\'o al mercado su \textbf{Nintendo Entertainment System (NES)} cuyo controlador sent\'o unas bases en el control del movimiento gracias a la cruceta que incorporaba. Esta cruceta permit\'ia un movimiento en 4 direcciones y que pretend\'ia reemplazar a las voluminosas palancas de mando de los controladores. Adem\'as de la cruceta, el mando dispon\'ia de 2 botones redondos (A y B) y otros 2 botones rectangulares (START y SELECT). En lo sucesivo, se lanzaron varios dispositivos especiales dise\~nados precisamente para usarse con juegos espec\'ificos, aunque muy pocos de estos se volvieron populares. Uno de estos dispositivos era el \textbf{Power Glove}, el que ser\'ia considerado como uno de los primeros perif\'ericos de interfaz en recrear los movimientos de la mano en una pantalla de televisi\'on o de un ordenador en tiempo real. En 1990 Nintendo hizo evolucionar a la Nintendo NES y lanz\'o la \textbf{Super Nintendo Entertainment System (SNES)}, la cual dej\'o atr\'as un dise\~no cuadrado del controlador y se inclin\'o por un dise\~no m\'as ergon\'omico, se mejor\'o la cruceta y se a\~nadieron otr\'os 2 botones (X e Y). El tiempo de respuesta del controlador era de 16 milisegundos. Dentro de la evoluci\'on de los controladores, en 1993 la compa\~nia SEGA sorprendi\'o con el lanzamiento de un nuevo accesorio para su consola \textbf{Sega Mega Drive}. Este accesorio consist\'ia en un aro octogonal que se colocaba en el sueloy se conectaba directamente al puerto de controlador de la consola. Lo llamaron \textbf{Sega Activator} y fue el primer controlador que utilizaba el cuerpo completo. El jugador se ten\'ia que situar en el centro del aro, el cual emit\'ia rayos infrarrojos hacia arriba para detectar los movimientos del jugador. Los juegos orientados para el Sega Activator eran juegos que involucrasen el movimiento de brazos y piernas para que el jugador cruzase los rayoss infrarrojos y as\'i se detectase el movimiento. Al tratarse de 8 segmentos, cada uno de estos segmentos estaba mapeado como si fuera un bot\'on en el mando tradicional, el cual se "pulsar\'ia" cada vez que el jugador cruzase un segmento de los rayos infrarrojos. \par

Durante los a\~nos 90 \textbf{Sony} entr\'o al terreno del desarrollo de consolas y por consecuencia, de modelos diferentes de controladores de videojuegos. Con su primera consola, la \textbf{Sony PlayStation}, incluyeron un nuevo dise\~no de mando que recog\'ia muchos de los dise\~nos vistos hasta el momento. A diferencia de Nintendo, este controlador cambi\'o la nomenclatura de los botones A,B,Y y X por las figuras $\triangle$, O, X y $\Box$, manten\'ia la cruceta y los botones START y SELECT y adem\'as a\~nadi\'o 4 botones m\'as en la parte lateral del mando para los dedos \'indice y coraz\'on. 3a\~nos m\'as tarde Sony sacar\'ia una re-edici\'on del mando al que le incorporaron 2 stickts anal\'ogicos junto con un bot\'on con un LED para cambiar entre los diferentes modos usados para el control del personaje. Este modelo fue el predecesor del famoso \textbf{DualShock} y \'unicamente la versi\'on japonesa presentaba una funci\'on de retroalimentaci\'on de vibraci\'on. Por el lado de Nintendo, la consola sucesora de la Super Nintendo fue la \textbf{Nintendo 64} que fue acompa\~nada por un nuevo dise\~no de mando que no pas\'o desapercibido. Dispon\'ia de una cruceta en la parte izquierda del mando, un stick de 360 grados y un bot\'on START en el centro del mando y 6 botones en su parte derecha. Complementario a esto, en la parte trasera del mando hab\'ia 2 botones m\'as y tambi\'en en la parte trasera se daba la opci\'on de introducir un dispositivo extraible que proporcionaba retroalimentaci\'on de vibraci\'on. Este accesorio se activaba en ocasiones concretas como al disparar un arma y serv\'ia para sumergir al jugador en el videojuego.\par

En los a\~nos posteriores las compa\~nias siguieron sacando modelos direferentes mandos que modificaban tama\~no y posiciones de los botones pero no salieron cambios significativos hasta que en 2002 Nintendo lanz\'o al mercado un nuevo mando alternativo para su consola\textbf{GameCube}, este controlador ten\'ia la peculiaridad de ser inal\'ambrico. Lo llamaron \textbf{WaveBird Wireless Controller} y sent\'o las bases para los pr\'oximos mandos inal\'ambricos. Contaba con una cruceta, 6 botones digitales, 2 botones h\'ibridos ya que hacian la funci\'on de gatillos y 2 palancas anal\'ogicas para el movimiento del personaje y la c\'amara normalmente. Como alimentaci\'on usaba 2 pilas AA y para comunicarse con la consola usaba radiofrecuencia, lo que permit\'ia al jugador alejarse hasta 6 metros de la consola. Poco tiempo despu\'es tanto Sony con su PlayStation 3 como Microsoft con su Xbox 360 a\~nadir\'ian las baterias a sus mandos para convertirlos en inal\'ambricos. \par

En 2006 Nintendo  volvi\'o a sorprender a la sociedad con la llegada de la \textbf{Nintendo Wii} y su nuevo mando.\textbf{Nintendo Wiimote} es el mando principal de la consola Wii y las caracter\'isticas destacables que trae son la de la detecci\'on de movimiento en el espacio y la habilidad de poder apuntar a objetos en la pantalla. El dise\~no del mando de Wii deja de lado todos los modelos tradicionales de mandos para videojuegos y se acerca m\'as a un control remoto de televisi\'on para que este pueda usarse con una sola mano y sea m\'as intuitivo ya que lo que pretend\'ia era cautivar a un p\'ublico m\'as casual y ser una consola para todos los miembros de la familia. En la cara frontal del mando se encuentran los botones "A", "1", "2", "+", "-", "HOME" y la cruceta, m\'as un bot\'on "POWER" para apagar la consola, algo in\'edito hasta entonces. En la parte anterior s\'olo presenta el bot\'on "B", en un formato similar a un gatillo. Adem\'as de los botones, en la parte frontal lleva incorporado un altavoz y 4 luces numeradas que indican el j\'umero de jugador al que corresponde cada mando durante una partida. Lleva una correa de seguridad udia al mando por laparte inferior de este para poder atar el mando a la mu\~neca y evitar asi que el mando se resbale durante una sesi\'on de juego y o el mando o la televisi\'on se vieran da\~nados. El Wii Remote tiene la capacidad de detectar la aceleraci\'on a lo largo de tres ejes mediante la utilizaci\'on de un aceler\'ometro. El Wiimote tambi\'en cuenta con un sensor \'optico PixArt, lo que le permite determinar el lugar al que el Wiimote est\'a apuntando; adem\'as de agregar una br\'ujula electr\'onica en su posterior versi\'on mejorada, el \textbf{WiiMotionPlus.} A diferencia de controladores que detectan la luz de la pantalla del televisor, el Wii Remote detecta la luz de una barra sensor que ven\'ia incorporada con la consola y se colocaba encima de la pantalla. No es necesario se\~nalar directamente a la barra sensor, pero apuntar significativamente fuera de la barra de posici\'on perturbar\'a la capacidad de detecci\'on debido al limitado \'angulo de visi\'on del Wiimote. La posici\'on y seguimiento del movimiento del Wii Remote permite al jugador imitar las acciones reales de juego, como blandir una espada o una pistola con objetivo, en lugar de simplemente pulsando los botones. El Wii Remote tiene incorporado un altavoz que emite sonidos diferentes que los emitidos por lo altavoces de la televisi\'on y que se utilizan para mejorar la ambientaci\'on del videojuego. El ejemplo de esto se mostr\'o en el E3 de 2006 cuando uno de los desarrolladores dispar\'o un arco en "The Legend of Zelda: Twilight Princess" y por el mando son\'o como tensaba y soltaba la flecha mientras que en la televisi\'on \'unicamente se escuchaba el impacto de la flecha. Esto daba la sensaci\'on de que la flecha viajaba desde el jugador hasta la televisi\'on. Esto se ve\'ia acompa\~nado por la vibraci\'on del mando y ambas funciones pod\'ian deshabilitarse desde el men\'u HOME de la consola, en ese caso todos los sonidos saldr\'ian por los altavoces de la televisi\'on. En cuanto a la duraci\'on de las baterias, con las funciones m\'as b\'asicas de algunos juegos ten\'ia la autonom\'ia de 60 horas y en caso de usar todas sus capacidades llegaba a aguantar 25 horas funcionando. Al tratarse de un mando sim\'etrico su uso estaba pensado tanto para zurdos como para diestros y se utilizaba tanto de manera vertical con una sola mano como de manera horizontal en caso de usar ambas manos. Adem\'as de todo lo anterior, en la parte inferior del mando se encuentra un puerto de expansi\'on para conectar diferentes perif\'ericos. Los m\'as usado fueron el \textbf{Wii Balance Board} que se trataba de una tabla capaz de calcular la presi\'on que se ejerc\'ia sobre ella, el \textbf{Wii Guitar} que permit\'ia jugar al juego Guitar Hero en su versi\'on de Wii, \textbf{Nunchuck} que agregaba un joystick anal\'ogico y 2 botones m\'as para el movimiento de personajes, \textbf{Wii Wheel} que hac\'ia que el Wii Remote se convirtiese en un volante y asi poder tener una mejor experiencia en juegos como Mario Kart y \textbf{Wii MotionPlus} que incorporaba 3 sensores nuevos de movimiento para los ejes vertical, longitudinal y lateral lo que ayud\'o a mejorar la precisi\'on del mando original.\par

En 2010 Microsoft dio el salto a un nuevo controlaor de videojuegos para su consola Xbox 360. Este nuevo perif\'erico es conocido con el nombre de \textbf{Kinect} y permite a los usuarios controlar e interactuar con la consola sin necesidad de tener contacto f\'isico con un mando tradicional. Este control se realiza por gestos y reconocimiento de voz. El sensor Kinect es una barra horizontal de unas 9 pulgadas conectada a una peque\~na base circular con un eje que permiteque esta rote y adem\'as est\'a dise\~nado para ser colocado por encima o por debajo de la televisi\'on. El dispositivo cuenta con una c\'amara RGB, un sensor de profundidad, un micr\'ofono de m\'ultiples matrices y un procesador personalizado que ejecuta el software patentado, que proporciona captura de movimiento de todo el cuerpo en 3D, reconocimiento facial y capacidades de reconocimiento de voz. El micr\'ofono de matrices del sensor de Kinect permite a la Xbox 360 llevar a cabo la localizaci\'on de la fuente ac\'ustica y la supresi\'on del ruido ambiente, permitiendo participar en el chat de Xbox Live sin utilizar auriculares. El sensor de profundidad es un proyector de infrarrojos combinado con un sensor CMOS monocromo que permite a Kinect ver la habitaci\'on en 3D en cualquier condici\'on de luz ambiental. El rango de detecci\'on de la profundidad del sensor es ajustable gracias al software de Kinect capaz de calibrar autom\'aticamente el sensor, basado en la jugabilidad y en el ambiente f\'isico del jugador, tal como la presencia de sof\'as, mesas y otro tipo de muebles.

Un 
Repaso de los diferentes dispositivos de entrada como son:
\begin {itemize}

\item Teclados
\item Ratones

\end {itemize}


Historia a parte de los diferentes mandos que hubo con datos relevantes que encuentre como frecuencia y dem\'as, eso es m\'as trabajo de dedicar tiempo a leer y buscar toda esta informaci\'on y plasmarla al final en una taba comparativa:
\begin {itemize}
\item recreativas
\item mandos cl\'asicos con conexi\'on por cable
\item mandos inal\'ambricos como los de PS3 y Xbox360 (utilizacion del bluetooth,infrarrojos, velocidad, latencia)
\item mandos con aceler\'ometros como wii y nintendo switch
\item Volantes y joysticks para simulaci\'on de vuelo: Explicaci\'on del force feedback, cuando se empieza a usar, implementaci\'on de este tipo de dispositivos en juegos y simuladores para el ej\'ercito.
\end {itemize}

Sistemas de streaming:
\begin {itemize}
\item qu\'e es un streaming de im\'agenes.
\item auge de este sistema con las retransmisiones en directo en plataformas como Youtube y Twitch (cifras de visualizaciones, monetizacion y dem\'as datos  que encuentre), TV por cable, Netflix y dem\'as servicios de streaming.
\item tabla con diferentes anchos de banda necesarios para diferentes configuraciones en las retrasnmisiones (720p, 1080p 30fps,1080p 60fps,etc)
\item est\'andares para la compresi\'on de video e im\'agen. 
\end {itemize}

%-------------------------------------------------------------------

%-------------------------------------------------------------------
\section{Herramientas para el desarrollo}
%-------------------------------------------------------------------
\label{cap2:sec:herramientas}

En este cap\'itulo se expondr\'an las diferentes tecnolog\'ias que se van a utilizar para la realizaci\'on de este proyecto. Adem\'as de eso, se pondr\'an encima de la mesa las diferentes aplicaciones desarrolladas por empresas del sector de los videojuegos que han sido tomadas como referencia para la creaci\'on de este trabajo. 
%-------------------------------------------------------------------
\subsection{Unity}
%-------------------------------------------------------------------
\label{cap2:subsec:unity}

Unity es un motor de videojuegos multiplaforma creado por Unity Technologies. Se encuentra disponible para sistemas Windows, Mac Os X y Linux. Es una de las herramientas de desarrollo de videojuegos m\'as populares actualmente en el mundo de los desarrolladores independientes. Con unity se han realizado algunos de los juegos m\'as famosos del mercado como \textbf{Ghost of a Tale}, \textbf{Cuphead} o  \textbf{Hollow Knight}.

Las principales caracter\'isticas buscadas son:

\begin{itemize}
\item \textbf{Multiplataforma.} El hecho de que Unity sea sistemas multiplataforma, te garantiza poder hacer juegos/aplicaciones que puedan ejecutarse en cualquier dispositivo a un coste bastante bajo en cuanto a esfuerzo.

\item  \textbf{Herramientas y servicios.} Unity es una herramienta que no engloba \'unicamente motores para el renderizado de im\'agenes, simulaciones f\'isicas 2D y 3D, animaci\'on de personajes y audio sino que al tratarse de un motor que usa C\#  como lenguaje de scripting, dispone de todas las herramientas de .NET para el desarrollo. Uno de los servicios que ofrece Unity es \textbf{Unity Analytics} que ayuda a los creadores a realizar anal\'iticas para ver c\'omo juegan sus jugadores.

\item  \textbf{Documentaci\'on.} Los desarrolladores de Unity ponen a disposici\'on de los usuarios una documentaci\'on amplia y detallada, tanto que incluso disponen de un historial de versiones de la documentaci\'on por si trabajas con un proyecto de Unity con una versi\'on no actualizada. Adem\'as de esto, ofrecen documentaci\'on sobre la API y ofrecen proyectos realizados por los propios desarrolladores del motor y tutoriales de como sacar el mayor rendimiento posible a las herramientas que ofrece el motor en su plataforma \textbf{Unity Learn.}.

\item  \textbf{Comunidad.} Debido a que Unity es un motor de videojuegos demandado y gratuito, cuenta con un gran n\'umero de desarrolladores que saben utilizarlo, lo que facilita la soluci\'on de problemas m\'as espec\'ificos. Unity tiene una comunidad muy amplia y muy activa tanto en foros como \textbf{Stack Overflow} y en su pripio portal de preguntas \textbf{Unity Answers.}.
\end{itemize}

%h	Establece la posición del elemento flotante «aquí». Ésto es, aproximadamente en el mismo punto donde aparece en el código (sin embargo, no siempre es exacto el posicionamiento)
%t	Inserta la figura al inicio de la página.
%b	Inserta la figura al final de la página.
%p	Inserta los elementos flotantes en una página por separado, que sólo contiene figuras.
%!	Sobreescribe los parámetros que LATEX usa para determinar una «buena» posición para la imagen.
%H	Establece el elemento flotante precisamente en el mismo lugar en el que aparece en el código, se requiere importar el paquete float. Es hasta cierto punto equivalente a h!.

%-------------------------------------------------------------------
\subsection{Android Studio}
%-------------------------------------------------------------------
\label{cap2:subsec:androidstudio}

Android Studio es el entorno de desarrollo oficial de la plataforma Android en contrapartida tambi\'en cabe la posibilidad de descargar las Android SDK sin necesidad del entorno completo de Android Studio. Fue desarrollada por Google y sustituy\'o a Eclipse en el desarrollo de aplicaciones para dispositivos Android. Las SDK de Android son imprescindibles para la creaci\'on de una aplicaci\'on Android, las SDK ofrecen un Emulador para poder ver tu aplicaci\'on en ejecuci\'on. Android Studio tiene la funci\'on de encapsular estas SDK de Android y poner en marcha la aplicaci\'on. Este IDE puede utilizarse tanto en Windows, Mac OS X y Linux pero \'unicamente puede usarse para el desarrollo de aplicaciones para Android.


Las principales caracter\'isticas de Android Studio son:

\begin{itemize}

\item \textbf{Espec\'ifico.} Si el producto que quieres desarrollar va a ser exclusivo de un sistema Android, con Android Studio te vas a centrar en explotar las funcionalidades de ese sistema al m\'aximo sin tener que lidiar con diferentes sistemas que no te interesan.
\item \textbf{Editor de dise'no.} Android Studio cuenta con un editor visual para poder acomodar el layout de tu aplicaci\'on Android de una manera mucho m\'as sencilla.

\end{itemize}
%-------------------------------------------------------------------
\subsection{ZXing}
%-------------------------------------------------------------------
\label{cap2:subsec:zxing}

Este es un proyecto del tipo Open-Source. Esta libreria de procesamiento de im\'agenes de c\'odigos de barras y QR's est\'a implementada en Java y tiene diferentes versiones en los distintos lenguajes. En el caso de este proyecto, se ha usado la versi\'on de .NET para usarla desde Unity. Al ser una librer\'ia que tiene soporte en muchos otros lenguajes que no son Java, puede ser utilizada en proyectos multiplataforma si estas plataformas usan diferentes leguajes como en este caso con Java y C\#.


\subsection{Android SDK}
%-------------------------------------------------------------------
\label{cap2:subsec:android}

Cada vez que Android saca una nueva versi\'on de su sistema operativo le acompa\~n su correspondiente SDK. Estas SDK de Android incluyen librerias que son necesarias para el desarrollo para dispositivos Android, adem\'as de documentaci\'on del API, un debugger y un emulador para probar los proyectos sin necesidad de tener un dispositivo f\'isico. Este SDK suele usarse acomplado a un IDE que como se ha explicado anteriormente en el punto \textbf{2.1.2} ha sido \textbf{Android Studio}.

%-------------------------------------------------------------------
\section{Interfaces de entrada}
%-------------------------------------------------------------------

\label{cap2:sec:proyectos-similares}
%-------------------------------------------------------------------

Hay empresas de videojuegos como Nintendo y Sony que han invertido mucho dinero en innovar y crear nuevas formas para que los usuarios disfruten de los diferentes juegos. Algunos de los ejemplos que ya existen en el mercado y han servido de inspiraci\'on para la realizaci\'on de este proyecto son:

\subsection{Wii U}
%-------------------------------------------------------------------
\label{cap2:subsec:Wii U}
%-------------------------------------------------------------------

Wii U es la consola dom\'estica que Nintendo cre\'o en la octava generaci\'on de consolas. La innovaci\'on principal de esta consola era la de cambiar radicalmente el modo de jugar a una consola de sobremesa ya que desarrollaron el Wii U GamePad. Este mando ten\'ia la funci\'on de una segunda pantalla y la de un mando tradicional a la vez. Esta pantalla port\'atil es t\'actil y recibe una se'nal de 480p.
\\
 Este nuevo mando permit\'ia a los desarrolladores tener un HUD mucho m\'as limpio dentro del juego ya que, en muchos casos, esta pantalla era utilizada para poner elementos como el minimapa y en algunos juegos como Mario Bros, hab\'ia cambios de zonas que convert\'ian al mando en pantalla principal. \\

\subsection{Controlador de videojuegos inal\'ambricos}
%-------------------------------------------------------------------
\label{cap2:subsec:GamePad}
%-------------------------------------------------------------------

Un controlador de videojuegps es un dispositivo de entrada cuya funci\'on es interactuar con los elementos de un juego para realizar diferentes acciones. Los controladores de videojuegos est\'an extendidos tanto en ordenadores como en consolas y en el \'ultimo a'no est\'an saliendo al mercado nuevos mandos para dispositivos m\'oviles, los cuales se conectan por bluetooth. Estos \'ultimos son mandos f\'isicos que no est\'an al alcance de todos y actualmente no son compatibles con todos los juegos, se utilizan para juegos muy concretos como PUBG Mobile. Los controladores de videojuegos han ido cambiando su forma y el n\'umero de botones de los que disponen. La necesidad del uso de mandos en videojuegos de m\'ovil hizo que Android 9 incluyese la posibilidad de hacer que un usuario conectase su Dualshock 3 a su Android para poder jugar.

\subsection{PlayLink}
%-------------------------------------------------------------------
\label{cap2:subsec:PlayLink}

PlayStation es una de las compa'n\'ias que m\'as tr\'afico de jugadores mueve llegando a la cifra de 90 millones de usuarios activos mensuales en enero de 2019, sus consolas son de las m\'as vendidas en todo el mundo y para que esto sea posible siempre tienen que intentar estar a la cabeza de nuevos perif\'ericos, nuevos juegos y, por supuesto, darles a sus usuarios las mejores experiencias posibles. En 2017, Sony PlayStation sac\'o al mercado una nueva serie de juegos llamados PlayLink. Estos juegos tienen una particularidad con respecto a un juego convencional de consola, estos juegos est\'an hechos para jugarlos con gente y usando un dispositivo m\'ovil como mando/controlador. Conectando la consola y el m\'ovil a la misma red WIFI y conect\'andolos a trav\'es de una aplicaci\'on, se pueden conectar de 2 a 8 jugadores, depende de los que admita cada juego, para poder jugar en familia o con amigos.

\subsection{PS4 Remote Play}
%-------------------------------------------------------------------
\label{cap2:subsec:ps4-remote-play}

En el \'ultimo trimestre del a\~no 2019, Sony lanz\'o la aplicaci\'on \textbf{PS4 Remote Play} para que los usuarios pudieran controlar su PlayStation 4 desde un dispositivo m\'ovil. Esta aplicaci\'on ten\'ia como requisito una conexi\'on a internet de entre 5 y 12 Mbps por conexi\'on LAN para una mejor experiencia. Esta aplicaci\'on permite no solamente el manejo de una PS4 usando un dispositivo m\'ovil sino que adem\'as permite conectar un Dualshock 4 para controlar el juego y usar el m\'ovil \'unicamente como pantalla.


% Variable local para emacs, para  que encuentre el fichero maestro de
% compilaci�n y funcionen mejor algunas teclas r�pidas de AucTeX
%%%
%%% Local Variables:
%%% mode: latex
%%% TeX-master: "../ManualTeXiS.tex"
%%% End:
