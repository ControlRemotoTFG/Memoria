%---------------------------------------------------------------------
%
%                          Cap�tulo 2
%
%---------------------------------------------------------------------
%
% 02EstructuraYGeneracion.tex
% Copyright 2009 Marco Antonio Gomez-Martin, Pedro Pablo Gomez-Martin
%
% This file belongs to the TeXiS manual, a LaTeX template for writting
% Thesis and other documents. The complete last TeXiS package can
% be obtained from http://gaia.fdi.ucm.es/projects/texis/
%
% Although the TeXiS template itself is distributed under the 
% conditions of the LaTeX Project Public License
% (http://www.latex-project.org/lppl.txt), the manual content
% uses the CC-BY-SA license that stays that you are free:
%
%    - to share & to copy, distribute and transmit the work
%    - to remix and to adapt the work
%
% under the following conditions:
%
%    - Attribution: you must attribute the work in the manner
%      specified by the author or licensor (but not in any way that
%      suggests that they endorse you or your use of the work).
%    - Share Alike: if you alter, transform, or build upon this
%      work, you may distribute the resulting work only under the
%      same, similar or a compatible license.
%
% The complete license is available in
% http://creativecommons.org/licenses/by-sa/3.0/legalcode
%
%---------------------------------------------------------------------

\chapter{Estado del arte}
\label{cap2}

\begin{FraseCelebre}
\begin{Frase}
  %La mejor estructura no garantizar� los resultados ni el rendimiento.
  %Pero la estructura equivocada es una garant�a de fracaso.
\end{Frase}
\begin{Fuente}
%Peter Drucker
\end{Fuente}
\end{FraseCelebre}

\begin{resumen}
 Este cap\'itulo explica las herramientas que han sido utilizadas en el desarrollo de este trabajo. Tambi\'en se ofrecen unos cuantos ejemplos en los cuales nos hemos basado y nos han llevado a pensar que la idea de este proyecto es algo \'util.
\end{resumen}

%-------------------------------------------------------------------
\section{Herramientas para el desarrollo}
%-------------------------------------------------------------------
\label{cap2:sec:herramientas}

En este proyecto se ha unificado el uso de varias tecnolog\'ias, en principio independientes, para formar una herramienta que una el desarrollo de videojuegos para ordenador y el uso de una aplicaci\'on de Android que nos facilite la conexi\'on del PC con Android. Estas tecnolog\'ias/herramientas han sido:
%-------------------------------------------------------------------
\subsection{Unity}
%-------------------------------------------------------------------
\label{cap2:subsec:unity}

Unity es un motor de videojuegos multiplaforma creado por Unity Technologies. Se encuentra disponible para sistemas Windows, Mac Os X y Linux. Es una de las herramientas de desarrollo de videojuegos m\'as populares actualmente en el mundo de los desarrolladores independientes. Tambi\'en cuenta con una gran cantidad de documentaci\'on generada tanto por los diversos usuarios como por sus creadores. Otra opci\'on barajada era Unreal Engine 4; este motor es m\'as potente que Unity y tambi\'en es altamente usado por las desarrolladoras.

Las principales caracter\'isticas buscadas son:

\begin{itemize}
\item \textbf{Multiplataforma.} El hecho de que Unity y Unreal sean sistemas multiplataforma, te garantiza poder hacer juegos/aplicaciones que puedan ejecutarse en cualquier dispositivo a un coste bastante bajo en cuanto a esfuerzo.

\item  \textbf{Potencia.} Unreal Engine 4 es m\'as potente que Unity gracias a su fotorrealismo y trabajo de gr\'aficos hiperrealista como en la demo de Star Wars creada por los desarrolladores del motor, aun asi Unity no es ,ni mucho menos, un motor mediocre ya que consigue buenos acabados gracias al uso de diferentes shaders y filtros. Lo decisivo es que Unity es un motor ya dado en la universidad que facilitaba el uso del mismo y ,que para el objeto de este proyecto, su potencia ya es m\'as que suficiente.

\item  \textbf{Documentaci\'on.} El caso de Unity en cuanto a documentaci\'on es algo que los desarroladores de Unity mantienen funcional de manera continua. Puedes acceder a la documentaci\'on de una manera muy intuitiva desde la misma p\'agina web de Unity y en los \'ultimos meses est\'an incluyendo proyectos de juegos completos de principio a fin.

\item  \textbf{Comunidad.} Debido a que Unity es un motor de videojuegos potente y gratuito, cuenta con un gran n\'umero de desarrolladores que saben utilizarlo, lo que facilita la soluci\'on de problemas m\'as espec\'ificos.
\end{itemize}

%h	Establece la posición del elemento flotante «aquí». Ésto es, aproximadamente en el mismo punto donde aparece en el código (sin embargo, no siempre es exacto el posicionamiento)
%t	Inserta la figura al inicio de la página.
%b	Inserta la figura al final de la página.
%p	Inserta los elementos flotantes en una página por separado, que sólo contiene figuras.
%!	Sobreescribe los parámetros que LATEX usa para determinar una «buena» posición para la imagen.
%H	Establece el elemento flotante precisamente en el mismo lugar en el que aparece en el código, se requiere importar el paquete float. Es hasta cierto punto equivalente a h!.

%-------------------------------------------------------------------
\subsection{Android Studio}
%-------------------------------------------------------------------
\label{cap2:subsec:androidstudio}

Android Studio es el entorno de desarrollo oficial de la plataforma Android en contrapartida tambi\'en cabe la posibilidad de descargar las Android SDK sin necesidad del entorno completo de Android Studio. Fue desarrollada por Google y sustituy\'o a Eclipse en el desarrollo de aplicaciones para dispositivos Android. Las SDK de Android son imprescindibles para la creaci\'on de una aplicaci\'on Android, las SDK ofrecen un Emulador para poder ver tu aplicaci\'on en ejecuci\'on. Android Studio tiene la funci\'on de encapsular estas SDK de Android y poner en marcha la aplicaci\'on. Este IDE puede utilizarse tanto en Windows, Mac OS X y Linux pero \'unicamente puede usarse para el desarrollo de aplicaciones para Android.


Las principales caracter\'isticas de Android Studio son:

\begin{itemize}

\item \textbf{Espec\'ifico.} Si el producto que quieres desarrollar va a ser exclusivo de un sistema Android, con Android Studio te vas a centrar en explotar las funcionalidades de ese sistema al m\'aximo sin tener que lidiar con diferentes sistemas que no te interesan.
\item \textbf{Editor de dise'no.} Android Studio cuenta con un editor visual para poder acomodar el layout de tu aplicaci\'on Android de una manera mucho m\'as sencilla.

\end{itemize}
%-------------------------------------------------------------------
\section{Librer\'ias usadas}
%-------------------------------------------------------------------
\subsection{ZXing}
%-------------------------------------------------------------------
\label{cap2:subsec:zxing}

Este es un proyecto del tipo Open-Source. Esta libreria de procesamiento de im\'agenes de c\'odigos de barras y QR's est\'a implementada en Java y tiene diferentes versiones en los distintos lenguajes. En nuestro caso hemos usando la versi\'on de .NET para usarla desde Unity.
Las principales caracter\'isticas de ZXing son:

\begin{itemize}

\item \textbf{Versatilidad.} Al ser una librer\'ia que tiene soporte en muchos otros lenguajes que no son Java, puedes utilizarla en tus proyectos multiplataforma si estas plataformas usan diferentes leguajes como en nuestro caso (Android Studio y Unity).

\end{itemize}
%-------------------------------------------------------------------
\section{Proyectos Similares}
%-------------------------------------------------------------------

\label{cap2:sec:proyectos-similares}
%-------------------------------------------------------------------

Hay algunas empresas que han invertido mucho en innovar y crear nuevas formas para que los usuarios disfruten de los diferentes juegos. Algunos de los ejemplos que ya existen en el mercado y han servido de inspiraci\'on para la realizaci\'on de este proyecto son:

\subsection{Wii U}
%-------------------------------------------------------------------
\label{cap2:subsec:Wii U}
%-------------------------------------------------------------------
Wii U es una consola de Nintendo de la octava generaci\'on. Esta consola de sobremesa inclu\'ia una pantalla port\'atil que serv\'ia a la vez de pantalla secundaria y mando. Esta pantalla port\'atil es t\'actil y recibe una se'nal de 480p. A este nuevo mando se le denomina Wii U GamePad. Este nuevo mando permit\'ia a los desarrolladores tener un HUD mucho m\'as limpio dentro del juego ya que, en muchos casos, esta pantalla era utilizada para poner elementos como el minimapa. \\

\subsection{Joystick}
%-------------------------------------------------------------------
\label{cap2:subsec:Joystick}
%-------------------------------------------------------------------
El joystick es un perif\'erico de entrada que consiste en una palanca que tiene libre movimiento sobre una base. Este informa sobre su \'angulo y direcci\'on al dispositivo al que est\'e conectado.
Los joysticks se han utilizado siempre en el mundo de los videojuegos, empezando por la Atari hasta llegar al DualShock de PlayStation que hay actualmente.  \\

\subsection{PlayLink}
%-------------------------------------------------------------------
\label{cap2:subsec:PlayLink}

PlayStation es una de las compa'n\'ias que m\'as tr\'afico de jugadores mueve llegando a la cifra de 90 millones de usuarios activos mensuales, sus consolas son de las m\'as vendidas en todo el mundo y para que esto sea posible siempre tienen que intentar estar a la cabeza de nuevos perif\'ericos, nuevos juegos y, por supuesto, darles a sus usuarios las mejores experiencias posibles. En 2017, Sony PlayStation sac\'o al mercado una nueva serie de juegos llamados PlayLink. Estos juegos tienen una particularidad con respecto a un juego convencional de consola, estos juegos est\'an hechos para jugarlos con gente y usando un dispositivo m\'ovil como mando/controlador. Conectando la consola y el m\'ovil a la misma red WIFI y conect\'andolos a trav\'es de una aplicaci\'on, se pueden conectar de 2 a 8 jugadores, depende de los que admita cada juego, para poder jugar en familia o con amigos.


% Variable local para emacs, para  que encuentre el fichero maestro de
% compilaci�n y funcionen mejor algunas teclas r�pidas de AucTeX
%%%
%%% Local Variables:
%%% mode: latex
%%% TeX-master: "../ManualTeXiS.tex"
%%% End:
