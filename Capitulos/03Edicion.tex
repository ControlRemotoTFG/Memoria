%---------------------------------------------------------------------
%
%                          Cap�tulo 3
%
%---------------------------------------------------------------------
%
% 03Edicion.tex
% Copyright 2009 Marco Antonio Gomez-Martin, Pedro Pablo Gomez-Martin
%
% This file belongs to the TeXiS manual, a LaTeX template for writting
% Thesis and other documents. The complete last TeXiS package can
% be obtained from http://gaia.fdi.ucm.es/projects/texis/
%
% Although the TeXiS template itself is distributed under the 
% conditions of the LaTeX Project Public License
% (http://www.latex-project.org/lppl.txt), the manual content
% uses the CC-BY-SA license that stays that you are free:
%
%    - to share & to copy, distribute and transmit the work
%    - to remix and to adapt the work
%
% under the following conditions:
%
%    - Attribution: you must attribute the work in the manner
%      specified by the author or licensor (but not in any way that
%      suggests that they endorse you or your use of the work).
%    - Share Alike: if you alter, transform, or build upon this
%      work, you may distribute the resulting work only under the
%      same, similar or a compatible license.
%
% The complete license is available in
% http://creativecommons.org/licenses/by-sa/3.0/legalcode
%
%---------------------------------------------------------------------

\chapter{Objetivos y especificaci\'on}
\label{cap3}
\label{cap:objetivos}

\begin{FraseCelebre}
\begin{Frase}
%Si quieres ser le�do m�s de una vez, no vaciles en borrar a menudo.
%Rem tene, verba sequentur (Si dominas el tema, las palabras vendr�n solas)
\end{Frase}
\begin{Fuente}
%Horacio
%Cat�n el Viejo
\end{Fuente}
\end{FraseCelebre}

\begin{resumen}
  Este cap\'itulo se centra en exponer los objetivos a cumplir por este proyecto, tales como la creaci\'on de la libreria (Unity), la apliaci\'on (Android) y la comunicaci\'on entre ambas ayudado por las metodolog\'ias y herramientas como Scrum y Github.
\end{resumen}

%-------------------------------------------------------------------
\section{Objetivos}
%-------------------------------------------------------------------
\label{cap3:sec:obejtivos}

Con este proyecto se han establecido los siguientes objetivos generales:

\begin{itemize}
\item Crear una librer\'ia para Unity3D que se comunique con una aplicaci\'on Android usando una comunicaci\'on Cliente-Servidor, Unity3D(Servidor), el cual se encargar\'a de recibir los inputs del Android(Cliente).
\item Crear la aplicaci\'on Android para usar el propio m\'ovil como mando virtual.
\item Crear un prototipo que demuestre las capacidades de la herramienta desarrollada.
\end{itemize}

A continuaci\'on plantearemos la metodolog\'ia y el plan de trabajo seguido.


%-------------------------------------------------------------------
\section{Plan de trabajo}
%-------------------------------------------------------------------
\label{cap3:sec:plandetrabajo}

La primera fase estar\'a dedicada a la investigaci\'on y el estudio de las herramientas ya existentes que exploran los aspectos en com\'un con este TFG. Se usar\'a Github como sistema de control de versiones, donde estar\'an 2 repositorios: uno para la memoria y otro donde se encontrar\'a el c\'odigo de la demo. El uso de Github es debido a su amplio reconocimiento a nivel mundial, en enero de 2013 ya contaba con 3 millones de usuarios junto con 4.9 millones de repositorios. Unos datos más actuales indican que Github en Junio de 2018 alojaba casi 80 millones de proyectos. Dado a la gran cantidad de usuarios de dicha plataforma existe una gran ayuda para todo problema que surja.
\\
Las reuniones con los directores ser\'an cada 2 semanas. El objetivo de estas reuniones ser\'a llevar un control de lo que se haya avanzado y plantear las pr\'oximas 2 semanas de trabajo.
Se han marcado unos hitos espec\'ificos:

 
\begin{enumerate}

\item \textbf{Navidad.} Sobre estas fechas deberemos tener la conexi\'on entre Android y Unity3D establecida. Con esto se pretende que un personaje en Unity3D sea capaz de moverse gracias al Input que recibe desde Android.
\item \textbf{Semana Santa.} Se deber\'a de haber conseguido tener una c\'amara en Unity3D enviando una imagen via streaming al dispositivo Android mientras se juega con este.
\item \textbf{Pre-Ex\'amenes Junio.} Aqui se debe tener una demo que sea capaz de dostrar las capacidades de la herramienta.

\end{enumerate}



%-------------------------------------------------------------------
\section{Metodolog\'ia}
%-------------------------------------------------------------------
\label{cap3:sec:metodolog\'ia}

La metodolog\'ia utilizada para la realizaci\'on de este proyecto est\'a basada en el desarrollo \'agil. Se han marcado peque'nos objetivos de 2 semanas de duraci\'on llamados sprints hasta llegar a las metas grandes o hitos mencionados en el apartado anterior.\\
La metodolog\'ia que se iba a usar estaba definida desde el primer d\'ia ya que es la que los miembros del proyecto siguen en todos los trabajos que realizan. Se decidi\'o usar \textbf{Scrum}.
Scrum es una metodolog\'ia \'agil que adopta una estrategia de desarrollo incremental, en lugar de la planificaci\'on y ejecuci\'on completa del proyecto en cuesti\'on. La particularidad que se tiene en este proyecto es el constante cambio de las tareas nuevas que se pueden llegar a incluir tras cada reuni\'on.

%-------------------------------------------------------------------
\section{Herramientas utilizadas}
%-------------------------------------------------------------------
\label{cap3:sec:herramientas}

En la secci\'on 3.1 se ha hablado de las herramientas que se acordaron utilizar para la realizaci\'on de este proyecto:

\begin{enumerate}

\item \textbf{MiKTeX y LaTeX}\\



 Las caracter\'isticas m\'as apreciables de MiKTeX son su habilidad de actualizarse por s\'i mismo descargando nuevas versiones de componentes y paquetes instalados previamente, y su f\'acil proceso de instalaci\'on. \\



\item \textbf{GitHub}\\



 GitHub es una plataforma de desarrollo cooperativo en la que se pueden alojar proyectos utilizando el sistema de control de versiones Git. Se utiliza para la creaci\'on y almacenamiento de c\'odigo fuente de manera p\'ublica. La herramienta Git se ha usado para mantener un control de versiones sobre 2 repositorios, uno para la demo y otro para la memoria. \\

\end{enumerate}

%\begin{table}[t]
%\footnotesize
%\centering
%\begin{tabular}{|l|c|c|}
%\hline
%Texto & Comando para \texttt{section} & Comando para �ndice \\
%\hline
%\hline
%Conclusiones & \verb+\Conclusiones+ & \verb+\TocConclusiones+ \\
%\hline
%En el pr�ximo cap�tulo\ldots & \verb+\ProximoCapitulo+ & \verb+\TocProximoCapitulo+ \\
%\hline
%Notas bibliogr�ficas & \verb+\NotasBibliograficas+ & \verb+\TocNotasBibliograficas+ \\
%\hline
%Resumen & \verb+\Resumen+ & \verb+\TocResumen+ \\
%\hline
%\end{tabular}
%\caption{Secciones no numeradas soportadas por \texis
  % \label{cap3:tab:seccionesnonumeradas}}
%\end{table}


% Variable local para emacs, para  que encuentre el fichero maestro de
% compilaci�n y funcionen mejor algunas teclas r�pidas de AucTeX
%%%
%%% Local Variables:
%%% mode: latex
%%% TeX-master: "../ManualTeXiS.tex"
%%% End:
