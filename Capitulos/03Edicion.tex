%---------------------------------------------------------------------
%
%                          Cap�tulo 3
%
%---------------------------------------------------------------------
%
% 03Edicion.tex
% Copyright 2009 Marco Antonio Gomez-Martin, Pedro Pablo Gomez-Martin
%
% This file belongs to the TeXiS manual, a LaTeX template for writting
% Thesis and other documents. The complete last TeXiS package can
% be obtained from http://gaia.fdi.ucm.es/projects/texis/
%
% Although the TeXiS template itself is distributed under the 
% conditions of the LaTeX Project Public License
% (http://www.latex-project.org/lppl.txt), the manual content
% uses the CC-BY-SA license that stays that you are free:
%
%    - to share & to copy, distribute and transmit the work
%    - to remix and to adapt the work
%
% under the following conditions:
%
%    - Attribution: you must attribute the work in the manner
%      specified by the author or licensor (but not in any way that
%      suggests that they endorse you or your use of the work).
%    - Share Alike: if you alter, transform, or build upon this
%      work, you may distribute the resulting work only under the
%      same, similar or a compatible license.
%
% The complete license is available in
% http://creativecommons.org/licenses/by-sa/3.0/legalcode
%
%---------------------------------------------------------------------

\chapter{Objetivos y especificaci\'on}
\label{cap3}
\label{cap:objetivos}

\begin{FraseCelebre}
\begin{Frase}
%Si quieres ser le�do m�s de una vez, no vaciles en borrar a menudo.
%Rem tene, verba sequentur (Si dominas el tema, las palabras vendr�n solas)
\end{Frase}
\begin{Fuente}
%Horacio
%Cat�n el Viejo
\end{Fuente}
\end{FraseCelebre}


%-------------------------------------------------------------------
\section{Objetivos}
%-------------------------------------------------------------------
\label{cap3:sec:obejtivos}

Tal y como se ha podido ver en los cap\'itulos anteriores, las empresas de videojuegos han puesto todos sus esfuerzos para adaptarse e integrar los dispositivos m\'oviles dentro del mundo de los videojuegos. Con las tecnolog\'ias mencionadas en el c\'apitulo dos como referencia directa, los objetivos que se plantearon para el proyecto son:


\begin{itemize}
\item Desarrollar una librer\'ia para Unity3D. Esta consistir\'a en un servidor  que ofrecer\'a su IP y el puerto elegido para la comunicaci\'on mediante un QR que aparecer\'a en pantalla.
\item Desarrollar una aplicaci\'on Android para usar el propio m\'ovil como mando virtual. Esta aplicaci\'on utilizar\'a la c\'amara para leer el c\'odigo QR e iniciar una conexi\'on con el servidor.
\item Integrar las herramientas mencionadas en los puntos anteriores en un proyecto ya cerrado para demostrar su funcionalidad.
\end{itemize}

Con la integraci\'on del proyecto en un juego ya cerrado se pretende hacer una serie de pruebas con usuarios para ver el rendimiento de la herramienta en ordenadores y dispositivos m\'oviles con diferentes caracter\'isticas. Para eso se implementar\'an herramientas para la recogida de datos de los usuarios y las acciones realizadas durante las sesiones. Estos datos ser\'an analizados en el cap\'itulo 5 con mayor profundidad.
 \\
A continuaci\'on plantearemos la metodolog\'ia y el plan de trabajo seguido.


%-------------------------------------------------------------------
\section{Plan de trabajo}
%-------------------------------------------------------------------
\label{cap3:sec:plandetrabajo}

La primera fase estar\'a dedicada a la investigaci\'on y el estudio de las herramientas ya existentes que exploran los aspectos en com\'un con este TFG. Se usar\'a Github\footnote{https://github.com/} como sistema de control de versiones, donde estar\'an 2 repositorios: uno para la memoria y otro donde se encontrar\'a el c\'odigo de la demo. El uso de Github es debido a su amplio reconocimiento a nivel mundial, en enero de 2013 ya contaba con 3 millones de usuarios junto con 4.9 millones de repositorios. Unos datos m\'as actuales indican que Github en Junio de 2018 alojaba casi 80 millones de proyectos. Dado a la gran cantidad de usuarios de dicha plataforma existe una gran ayuda para todo problema que surja.
\\
Las reuniones con los directores ser\'an cada 2-3 semanas, dependiendo de la disponibilidad y horarios. El objetivo de estas reuniones ser\'a llevar un control de lo que se haya avanzado y plantear las pr\'oximas 2-3 semanas de trabajo.
Se han marcado unos hitos espec\'ificos:

 
\begin{enumerate}

\item \textbf{Hito 1:} Conexi\'on entre Android y Unity3D establecida. Con esto se pretende que un personaje en Unity3D sea capaz de moverse gracias al Input que recibe desde Android.
\item \textbf{Hito 2:} Se deber\'a de haber conseguido tener una c\'amara en Unity3D enviando una imagen via streaming al dispositivo Android mientras se juega con el Input recibido del dispositivo m\'ovil.
\item \textbf{Hito 3:} Aqui se debe tener una demo que sea capaz de demostrar las capacidades de la herramienta.
\item \textbf{Hito 4:} Pruebas con usuarios, recogida de informaci\'on y an\'alisis.
\end{enumerate}



%-------------------------------------------------------------------
\section{Metodolog\'ia}
%-------------------------------------------------------------------
\label{cap3:sec:metodolog\'ia}

La metodolog\'ia utilizada para la realizaci\'on de este proyecto est\'a basada en el desarrollo \'agil. Se han marcado peque'nos objetivos de 2 semanas de duraci\'on llamados sprints hasta llegar a las metas grandes o hitos mencionados en el apartado anterior.\\
La metodolog\'ia que se iba a usar estaba definida desde el primer d\'ia ya que es la que los miembros del proyecto siguen en todos los trabajos que realizan. Se decidi\'o usar \textbf{Scrum}.
Scrum es una metodolog\'ia \'agil que adopta una estrategia de desarrollo incremental, en lugar de la planificaci\'on y ejecuci\'on completa del proyecto en cuesti\'on. La particularidad que se tiene en este proyecto es el constante cambio de las tareas nuevas que se pueden llegar a incluir tras cada reuni\'on.

%-------------------------------------------------------------------
\section{Herramientas utilizadas}
%-------------------------------------------------------------------
\label{cap3:sec:herramientas}

Es resultado final del proyecto involucra tanto un ordenador como un dispositivo m\'ovil conectados a la misma red. Para el desarrollo usaremos las siguientes herramientas software de comunicaci\'on, seguimiento y planificaci\'on:

\begin{enumerate}

\item \textbf{\LaTeX \footnote{https://www.latex-project.org/}}\\

Para la realizaci\'on de este documento se ha usado \LaTeX en su distribuci\'on de MiKTeX \footnote{https://miktex.org/} para Windows.
Las caracter\'isticas m\'as apreciables de MiKTeX son su habilidad de actualizarse por s\'i mismo descargando nuevas versiones de componentes y paquetes instalados previamente, y su f\'acil proceso de instalaci\'on.\\

\item \textbf{GitHub}\\

 GitHub es una plataforma de desarrollo cooperativo en la que se pueden alojar proyectos utilizando el sistema de control de versiones Git\footnote{https://git-scm.com/}. Se utiliza para la creaci\'on y almacenamiento de c\'odigo fuente de manera p\'ublica. La herramienta Git se ha usado para mantener un control de versiones sobre 2 repositorios, uno para la demo y otro para la memoria. \\

\item \textbf{Discord\footnote{https://discord.com/}}\\

Discord es una plataforma de comunicaci\'on por voz, texto y v\'ideo. Discord ofrece la posibilidad de crear canales dedicados, lo que se ha usado para guardar enlaces de inter\'es para el desarrollo prematuro de este proyecto. 
Esta herramienta es multiplataforma y no requiere de instalaci\'on ya que puede usarse desde navegadores como Chrome o Firefox. Adem\'as de esto consta con un sistema de tranferencia de archivos y de retransmisi\'on de tu pantalla a las personas que se encuentren en la llamada. Todas estas caracter\'isticas han sido utilizadas para las sesiones de trabajo grupal. \\
\\

\item \textbf{Unity\footnote{https://unity.com/es}}\\

Unity ha sido el entorno de desarrollo elegido, la versi\'on de Unity utilizada ha sido la 2018.4.18f1. La elecci\'on de Unity fue por la gran comunidad de usuarios y la extensa y detallada documentaci\'on con la que cuenta este motor. Adem\'as, al usar C\# como lenguaje de programaci\'on, se ha contado con toda la API y documentaci\'on de Microsoft sobre .NET.\footnote{https://docs.microsoft.com/es-es/dotnet/} 

\item \textbf{Visual Studio 2019\footnote{https://visualstudio.microsoft.com/es/}}\\

La decisi\'on de usar Visual Studio como IDE fue por la integraci\'on que tiene con Unity. Gracias a esta integraci\'on se ha podido crear y mantener archivos de proyectos de Visual Studio autom\'aticamente. Algo a tener en cuenta es que no se usa el compilador de Visual Studio, sino que se usa directamente el compilador de C\# para la compilaci\'on de los scripts creados en Unity.\\

\item \textbf{Android Studio\footnote{https://developer.android.com/studio}}\\

Se ha decidido usar este IDE para el desarrollo de la aplicaci\'on para Android. Android Studio dispone de una construcci\'on del proyecto basada en Gradle\footnote{https://gradle.org/}, lo que hace que su construcci\'on se haga de manera din\'amica y un editor de dise\~no enriquecido que permite a los usuarios arrastrar y soltar componentes de la interfaz de usuario.
Adem\'as de incluir las SDK de Android y todo lo que estas incorporan para que el desarrollo en Android sea posible.
\\
\end{enumerate}

%\begin{table}[t]
%\footnotesize
%\centering
%\begin{tabular}{|l|c|c|}
%\hline
%Texto & Comando para \texttt{section} & Comando para �ndice \\
%\hline
%\hline
%Conclusiones & \verb+\Conclusiones+ & \verb+\TocConclusiones+ \\
%\hline
%En el pr�ximo cap�tulo\ldots & \verb+\ProximoCapitulo+ & \verb+\TocProximoCapitulo+ \\
%\hline
%Notas bibliogr�ficas & \verb+\NotasBibliograficas+ & \verb+\TocNotasBibliograficas+ \\
%\hline
%Resumen & \verb+\Resumen+ & \verb+\TocResumen+ \\
%\hline
%\end{tabular}
%\caption{Secciones no numeradas soportadas por \texis
  % \label{cap3:tab:seccionesnonumeradas}}
%\end{table}


% Variable local para emacs, para  que encuentre el fichero maestro de
% compilaci�n y funcionen mejor algunas teclas r�pidas de AucTeX
%%%
%%% Local Variables:
%%% mode: latex
%%% TeX-master: "../ManualTeXiS.tex"
%%% End:
