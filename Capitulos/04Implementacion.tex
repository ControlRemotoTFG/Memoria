%---------------------------------------------------------------------
%
%                          Cap�tulo 4
%
%---------------------------------------------------------------------
%
% 04Imagenes.tex
% Copyright 2009 Marco Antonio Gomez-Martin, Pedro Pablo Gomez-Martin
%
% This file belongs to the TeXiS manual, a LaTeX template for writting
% Thesis and other documents. The complete last TeXiS package can
% be obtained from http://gaia.fdi.ucm.es/projects/texis/
%
% Although the TeXiS template itself is distributed under the 
% conditions of the LaTeX Project Public License
% (http://www.latex-project.org/lppl.txt), the manual content
% uses the CC-BY-SA license that stays that you are free:
%
%    - to share & to copy, distribute and transmit the work
%    - to remix and to adapt the work
%
% under the following conditions:
%
%    - Attribution: you must attribute the work in the manner
%      specified by the author or licensor (but not in any way that
%      suggests that they endorse you or your use of the work).
%    - Share Alike: if you alter, transform, or build upon this
%      work, you may distribute the resulting work only under the
%      same, similar or a compatible license.
%
% The complete license is available in
% http://creativecommons.org/licenses/by-sa/3.0/legalcode
%
%---------------------------------------------------------------------

\chapter{Implementaci\'on de las aplicaciones}
\label{cap4}
\label{cap:impl}

En el cap\'itulo anterior se analizaron las diferentes opciones de dise\~no del proyecto y se detall\'o el protocolo de comunicaci\'on entre el juego y el dispositivo m\'ovil. Adem\'as de esto se expusieron las ventajas y desventajas de usar los diferentes conjuntos de caracter\'isticas expuestas. En este cap\'itulo se detallar\'a todo el proceso de creaci\'on y desarrollo de las diferentes aplicaciones que utilizan el protocolo de comunicaci\'on anteriormente expuesto.
Se ha divido este cap\'itulo en 2 secciones, una para cada aplicaci\'on desarrollada:

\begin {itemize}
\item Implementaci\'on de la aplicaci\'on de Android (secci\'on \ref{android})
\item Implementaci\'on de la aplicaci\'on de Unity (secci\'on \ref{unity})
\end {itemize}

%-------------------------------------------------------------------
\section{Implementaci\'on de la aplicaci\'on de Android}
\label{android}
%-------------------------------------------------------------------

Android Studio es el entorno de desarrollo integrado (IDE) oficial para la plataforma de Android. Hasta finales del 2014 para desarrollar en Android se utilizaba Eclipse como IDE oficial. Actualmente Android Studio est\'a disponible para las plataformas Microsoft Windows, macOS y GNU/Linux. Android Studio incluye una gran variedad de herramientas para facilitar el desarrollo en Android entre las que se incluyen:

\begin {itemize}
\item Plantillas para crear dise\~nos comunes de Android y otros componentes.
\item Un editor de dise\~no enriquecido que permite a los usuarios arrastrar y soltar componentes de la interfaz de usuario.
\item Soporte para construcci\'on basada en Gradle.
\item Dispositivos virtuales de Android que se utiliza para ejecutar y probar aplicaciones.
\item Consola de desarrollador: consejos de optimizaci\'on, ayuda para la traducci\'on y estad\'isticas de uso.
\item Integraci\'on de ProGuard y funciones de firma de aplicaciones.
\end {itemize}
 
Los lenguajes de programaci\'on aceptados por Android Studio son Kotlin, Java y C++. En concreto para este proyecto se ha usado Java como lenguaje de programaci\'on.

%-------------------------------------------------------------------
\subsection {Conceptos b\'asicos de Android}
%-------------------------------------------------------------------

Las aplicaciones en Android se rigen por una serie de componentes que se llaman \textbf{\textit{Activity}}. Estos componentes son claves a la hora de manejar los estados de una aplicaci\'on en Android. A diferencia de otros paradigmas de programaci\'on que comienzan sus aplicaciones con un m\'etodo \textbf{\textit{main()}}, la instancia de una Actividad invoca m\'etodos de devoluci\'on de llamada que se corresponden con etapas espec\'ificas de su ciclo de vida. La experiencia de una aplicaci\'on para un dispositivo m\'ovil difiere mucho de la versi\'on de escritorio de esa misma aplicaci\'on ya que la interacci\'on del usuario con la aplicaci\'on no siempre comienza en el mismo lugar. Un claro ejemplo de esto sucede con las aplicaciones de mensajer\'ia instantanea. Un usuario puede estar navegando por cualquier red social y encontrarse una publicaci\'on interesante y compartirla por correo electr\'onico o por una aplicaci\'on de mensajer\'ia instantanea. La aplicaci\'on de correo electr\'onico no se abre en el mismo estado si se abre desde la opci\'on de compartir de la red social o si se abre desde el men\'u de aplicaciones instaladas en el dispositivo. Las Actividades est\'an dise\~nadas para facilitar este paradigma.

\begin{figure}[!htb]
    \centering
    \includegraphics[width=0.90\textwidth]{./Imagenes/Vectorial/lifecycle-states.pdf}
    \caption{Estados de las aplicaciones Android}
\label{Fig:estados}
\end{figure}

La mayor\'ia de las aplicaciones contienen varias pantallas, lo cual significa que contienen varias actividades. Este concepto se pondr\'a posteriormente de manifiesto ya que para el desarrollo de la aplicaci\'on han sido necesarias 2 Actividades.
Cuando un usuario navega por una aplicaci\'on, la cierra, la vuelve a abrir o la minimiza, las instancias de las Actividades de la aplicaci\'on pasan por una serie de estados de su ciclo de vida (figura \ref{Fig:estados}). Estos estados pueden tener comportamientos definidos por los desarrolladores de la aplicaci\'on. Esto permite tener un control sobre lo que ocurre en cada uno de los estados de la aplicaci\'on. Para navegar por las transiciones entre las etapas del ciclo de vida de una actividad, la clase Activity proporciona un conjunto b\'asico de seis devoluciones de llamadas: \textbf{onCreate(), onStart(), onResume(), onPause(), onStop() y onDestroy().} El sistema invoca cada una de estas devoluciones de llamada cuando una actividad entra en un nuevo estado.


\begin{figure}[h]

\centering
\includegraphics[width=0.7\textwidth]{./Imagenes/Bitmap/Ciclo_de_vida_Android}
\caption{Ciclo de vida de una Actividad de un sistema Android}
\end{figure}


Cada uno de los estados que tiene la actividad es llamado en un momento concreto de la ejecuci\'on de la Actividad. Las caracter\'isticas de cada uno de estos estados son las siguientes:

\begin {itemize}
\item \textbf{onCreate()} $\rightarrow$ Este m\'etodo es el primero que llama cuando se crea la Actividad. Este estado es utilizado para ejecutar la l\'ogica de la aplicaci\'on que debe ocurrir \'unicamente una vez en todo el ciclo de vida. Este m\'etodo recibe el par\'ametro \textit{\textbf{savedInstanceState}}, que es un objeto de tipo \textbf{\textit{Bundle}} que contiene el estado ya guardado de la actividad. Si la actividad nunca existi\'o, el valor del objeto \textit{Bundle} es nulo.
\item \textbf{onStart()} $\rightarrow$ Cuando la actividad entra en el estado Started, el sistema invoca esta devoluci\'on de llamada. La llamada \textit{onStart()} hace que el usuario pueda ver la actividad mientras la app se prepara para que esta entre en primer plano y se convierta en interactiva.
\item \textbf{onResume()} $\rightarrow$  La aplicaci\'on permanece en el estado \textit{} hasta que ocurre alg\'un evento que la quita de foco. Tal evento podr\'ia ser, por ejemplo, recibir una llamada telef\'onica, que el usuario navegue a otra actividad o que se apague la pantalla del dispositivo.
\item \textbf{onPause()} $\rightarrow$ Este estado se utiliza cuando se ha perdido el foco de una aplicaci\'on. Sin embargo una actividad con el estado Paused puede ser completamente visible si est\'a en el modo multiventana. En este estado no deben guardarse datos de la aplicaci\'on ya que es un estado que dura poco tiempo.
\item \textbf{onStop()} $\rightarrow$ En este estado es donde los componentes del ciclo de vida pueden detener cualquier funcionalidad que no necesite ejecutarse mientras el componente no sea visible en la pantalla. Este estado debe usarse para liberar o ajustar recursos que no son necesarios mientras no sea visible para el usuario.
\item \textbf{onDestroy()} $\rightarrow$ Se llama a este m\'etodo antes de que se finalice la actividad. El sistema invoca esta devoluci\'on de llamada cuando la aplicaci\'on se cierra. En este estado es donde los componentes del ciclo de vida pueden recuperar cualquier elemento que se necesite antes de que finalice la Actividad.
\end {itemize}

En el siguiente apartado se explicar\'a el uso de estas llamadas del sistema Android dentro del proyecto y la utilizaci\'on de diferentes actividades. Adem\'as se expondra la arquitectura de la aplicaci\'on desarrollada para Android.


%-------------------------------------------------------------------
\subsection {Desarrollo de la aplicaci\'on Android}
%-------------------------------------------------------------------

Una vez se tienen claras las diferentes llamadas que realiza el sistema Android y cu\'ando las realiza, es el momento de ver la aplicaci\'on que tienen en el desarrollo de este proyecto. Tal y como se plante\'o en la descripci\'on del protocolo de comunicaci\'on, la aplicaci\'on de Android necesita saber la IP y el puerto al que debe conectarse para enviar las pulsaciones. Para este primer paso se decidi\'o utilizar un c\'odigo QR que al leerlo con el m\'ovil, este contenga la direcci\'on IP y el puerto al que la aplicaci\'on se debe conectar. La lectura del c\'odigo QR y el env\'io de las puslaciones en pantalla son 2 procesos muy distintos e independientes, por esta raz\'on se implementaron 2 actividades diferentes.\\

La primera de estas actividades tiene como funci\'on activar la c\'amara del dispositivo Android para leer un c\'odigo QR. El contenido de este c\'odigo QR viene dado con el formato IP:Puerto (por ejemplo 192.168.1.1:5000). Una vez el c\'odigo QR es correcto, Android ofrece la posibilidad de lanzar una actividad nueva con una serie de par\'ametros. Para este proyecto uno de estos par\'ametros ser\'a el contenido del c\'odigo QR para que sea la actividad hija la que realice la conexi\'on y el intercambio de datos con el videojuego. \\

En esta segunda actividad se realiza la conexi\'on con el videojuego, lo que implica el env\'io de pulsaciones y la recepci\'on de im\'agenes y mensajes de vibraci\'on. Para que estos procesos se realicen de manera independiente y as\'incrona se han implementado 2 hilos de ejecuci\'on diferentes. Uno de ellos se encarga de gestionar la recepci\'on de datos y otro del env\'io de pulsaciones. Como esto debe realizarse al inicio de la ejecuci\'on de la activididad, el proceso de inicializaci\'on de estos hilos junto con los \textit{listeners} de las pulsaciones de la pantalla se realizan en el m\'etodo \textit{onCreate()}. \\

La actividad llamar\'a a la funci\'on \textit{onPause()} en caso de que se pierda el foco de la aplicaci\'on, es por eso que para parar por completo el consumo de la aplicaci\'on se cierran los hilos y se manda el mensaje de cierre de conexi\'on al juego. Esto ocurre para que no sea el sistema Android el que cierre los hilos de una manera inesperada.\\

Para que la aplicaci\'on Android cumpla los requisitos expuestos en el apartado de especificaci\'on del proyecto, se han implementado 4 clases:

\begin {itemize}
\item \textbf{MainActivity} $\rightarrow$ Esta primera clase corresponde a la primera actividad. En esta primera actividad se utiliza la c\'amara para leer un c\'odigo QR y guardar como par\'ametro el contenido del c\'odigo. Este contenido es la IP del equipo donde se est\'a ejecutando el juego y el puerto al que deben ser enviadas las pulsaciones del usuario. Estos datos son pasados a la actividad hija para comenzar a usar el m\'ovil como dispositivo de entrada.
\item \textbf{Controller} $\rightarrow$ Esta es la actividad principal. Desde esta actividad se recogen los datos de IP y puerto al que debe conectarse el dispositivo Android gracias a la lectura de un c\'odigo QR realizada con la actividad padre. Al crearse la actividad se lanza la ejecuci\'on de una hebra que se encargar\'a de recibir, leer e interpretar las im\'agenes que lleguen por red una vez la aplicaci\'on se conecte al juego. Desde esta clase se controla la pulsaci\'on del usuario en la pantalla. De esta pulsaci\'on se guardan 3 datos: posici\'on (x,y) donde se ha realizado la pulsaci\'on y el tipo de pulsaci\'on (presionar, levantar o arrastrar). Este dato permite soportar el \textit{multitouch} en la aplicaci\'on. El comportamiento de este hilo viene dado en la clase \textbf{UdpClientThread} que se encarga de preparar el datagrama de la pulsaci\'on y enviarlo. La \'ultima funci\'on de esta clase es la de cambiar la imagen que se muestra en la aplicaci\'on. 

\item \textbf{UdpClientThread} $\rightarrow$ Esta clase tiene como funci\'on el env\'io de datos a la aplicaci\'on donde se ejecuta el juego.Esta hebra se encarga del env\'io de paquetes que incluyen el tipo de pulsaci\'on que se ha realizado, la coordenada \textit{x} y la coordenada \textit{y} de la pantalla del dispositivo donde se ha realizado la pulsaci\'on. Una vez que la aplicaci\'on se cierre, el datagrama de cierre de conexi\'on se env\'ia desde esta hebra.

\item \textbf{Receive\_Image} $\rightarrow$ Esta clase se ejecuta desde una hebra distinta a la de la Actividad principal y la funci\'on que desempe\~na es la de recibir informaci\'on que mande el juego.En este hilo se espera la llegada del \textit{streaming} de im\'agenes desde el juevo. Estas im\'agenes se esperan en formato PNG ya que se realiza la descompresi\'on de este formato. El tiempo de vibraci\'on puede ser modificado en cualquier momento por el desarrollador por lo que ese mensaje tambi\'en es tratado en este hilo.
\end {itemize}


\begin{figure}[h]

\centering
\includegraphics[width=0.9\textwidth]{./Imagenes/Vectorial/Arquitectura_App_Android.pdf}
\caption{Diagrama de clases de la aplicaci\'on Android}
\end{figure}

Se ha optado por una aplicaci\'on cerrada para que el desarrollador no tenga que implementar nueva funcionalidad en Android. El coste computacional de la aplicaci\'on es bajo, lo que permite ser utilizado en una gran cantidad de dispositivos. La divisi\'on en 2 actividades es debido a que el flujo de Android se basa en ir cambiando entre actividades para que cada una tenga un uso espec\'ifico. La primera actividad de usa para capturar un QR con la c\'amara y la segunda es utilizada para simular un mando.

%-------------------------------------------------------------------
\section{Implementaci\'on de la aplicaci\'on de Unity}
\label{unity}
%-------------------------------------------------------------------

Como se coment\'o al principio de este cap\'itulo, el motor de videojuegos elegido para la realizaci\'on de este proyecto ha sido Unity. Unity es un motor de videojuegos multiplataforma creado por \textit{Unity Technologies} en 2005. Unity est\'a disponible como plataforma de desarrollo para Microsoft, Mac OS y Linux y tiene soporte de compilaci\'on con m\'ultiples plataformas:

\begin {itemize}
\item \textbf{Web} $\rightarrow$ WebGL.
\item \textbf{PC} $\rightarrow$ Windows, SteamOS, Linux, OS X y Windows Store Apps.
\item \textbf{Dispositivos m\'oviles} $\rightarrow$ iOS, Android, Windows Phone.
\item \textbf{Smart TV} $\rightarrow$ tvOS, Samsung Smart TV, Android TV.
\item \textbf{Consolas} $\rightarrow$ PlayStation Vita, PlayStation 4, Xbox 360, Xbox One, Wii U, Nintendo 3DS, Nintendo Switch.
\item \textbf{Dispositivos de realidad virtual} $\rightarrow$ Oculus Rift, Google Cardboard, HTC Vive, PlayStation VR, Samsung Gear VR
\end {itemize}

Actualmente en la versi\'on 2021.1 de la documentaci\'on de Unity, no existe soporte para la nueva generaci\'on de consolas.\\


%-------------------------------------------------------------------
\subsection {Funcionamiento de Unity}
%-------------------------------------------------------------------

Unity es un motor de videojuegos que aglutina una gran variedad de herramientas para el desarrollo. Estas herramientas van desde inclusiones de \textbf{Scripts} para dar comportamientos espec\'ificos a cada una de las \textbf{Entidades} del juego hasta elementos m\'as visuales como diagramas de estado para el control de las animaciones de un modelo. Para que todos estos sistemas tan diferentes puedan convivir, hay una serie de funciones que se ejecutan en un orden determinado. Unity a su vez se compone de varios elementos clave:

\begin {itemize}
\item \textbf{Escena} $\rightarrow$  Las escenas contienen los objetos del juego. Pueden usarse para crear niveles, men\'us o cualquier estado del juego.
\item \textbf{GameObjects / Entidades} $\rightarrow$ Cada una de las escenas contiene objetos. Estos objetos se llaman GameObjects. Cualquier elemento es considerado un GameObject, no tiene por qu\'e tener una representaci\'on visual (m\'usica, c\'amara, etc).
\item \textbf{Componentes} $\rightarrow$ Los componentes son los diferentes atributos que se le dan a los GameObjects para que tengan funcionalidad (movimiento, posici\'on, animaci\'on, colisi\'on f\'isica, etc).
\end {itemize}

Unity ofrece una serie de componentes que dan una funcionalidad ya definida a un objeto, esta funcionalidad va desde tener una posici\'on definida en el mundo hasta emitir un sonido y realizar una animaci\'on. Los desarrolladores pueden desarrollar sus propios componentes usando Scipts. Estos scripts indican a las diferentes entidades c\'omo comportarse. El lenguaje seleccionado para este sistema de \textit{scripting} es C\# y un script debe estar vinculado a una entidad para que este se ejecute. \\

%-------------------------------------------------------------------
\subsection {Desarrollo de la API en Unity}
%-------------------------------------------------------------------

Las caracter\'isticas espec\'ificas de Unity deben tenerse en cuenta para la integraci\'on de la librer\'ia dentro del motor pero la librer\'ia est\'a desarrollada en .NET. El inicio de la comunicaci\'on entre en juego y  el m\'ovil se realiza con la lectura de un QR que lleva los datos de IP del PC donde se est\'a ejecutando el juego y el puerto donde el juego va a estar escuchando y por donde llegar\'an los datos del m\'ovil. Para conseguir que el m\'ovil disponga de estos datos se ha decidido utilizar un c\'odigo QR. Este c\'odigo se genera utilizando la librer\'ia \textbf{ZXing}\footnote{https://archive.codeplex.com/?p=zxingnet} en su versi\'on de .NET. \\

Para que la aplicaci\'on desarrollada en Unity cumpla los requisitos expuestos en el apartado de especificaci\'on del proyecto, se han realizado 2 clases:

\begin {itemize}
\item \textbf{UDPSocket} $\rightarrow$ Esta clase se utiliza para la creaci\'on de todo lo necesario para hacer funcionar esta herramienta. Con el m\'etodo \textbf{init()} se inician 2 hebras de ejecuci\'on diferentes. Una de ellas se encarga de enviar los datos necesarios al m\'ovil. Estos datos son tanto la vibraci\'on como la imagen a renderizar en el dispositivo o los mensajes de tipo \textit{keepalive} en caso de que el usuario no interactue con el juego en un tiempo determinado. La otra se encarga de recibir los datos de entrada del dispositivo y avisar a los diferentes \textit{listeners}. Estos listeners utilizan esa informaci\'on para los prop\'ositos desigandos por el desarrollador del juego (mover al personaje, pausar el juego, salir, etc). Esta clase tambi\'en se encarga de cerrar la conexi\'on.

\item \textbf{InputMobileInterface} $\rightarrow$ Esta interfaz permite al desarrollador del juego recibir los eventos que llegan desde el m\'ovil. Estos eventos son las coordenadas de las pulsaciones, las dimensiones del m\'ovil y el aviso del cierre de la conexi\'on por parte del m\'ovil. El m\'etodo \textbf{EndOfConnection()} comunica la p\'erdida de conexi\'on al \textit{listener} para que el usuario de la librer\'ia realice una acci\'on, por ejemplo pausar el juego hasta que la conexi\'on vuelva a iniciarse.
\end {itemize}

\begin{figure}[h]

\centering
\includegraphics[width=0.9\textwidth]{./Imagenes/Vectorial/Arquitectura Unity.pdf}
\caption{Diagrama de clases de la aplicaci\'on de Unity}
\end{figure}

En este cap\'itulo se ha expuesto el desarrollo de las aplicaciones de Android y Unity. Adem\'as se han explicado algunos de los conocimientos b\'asicos para usar ambas herramientas. En el pr\'oximo cap\'itulo se explicar\'a la integraci\'on de la librer\'ia en un juego ya terminado con el que se realizar\'an una serie de pruebas con usuarios. Estas pruebas servir\'an para obtener datos del rendimiento del proyecto. Aplicando una serie de baremos se determinar\'a si el uso de la librer\'ia desarrollada cumple con las espectativas.\\

% Variable local para emacs, para  que encuentre el fichero maestro de
% compilaci�n y funcionen mejor algunas teclas r�pidas de AucTeX
%%%
%%% Local Variables:
%%% mode: latex
%%% TeX-master: "../ManualTeXiS.tex"
%%% End:
