%---------------------------------------------------------------------
%
%                          Cap�tulo 6
%
%---------------------------------------------------------------------
%
% 05Bibliografia.tex
% Copyright 2009 Marco Antonio Gomez-Martin, Pedro Pablo Gomez-Martin
%
% This file belongs to the TeXiS manual, a LaTeX template for writting
% Thesis and other documents. The complete last TeXiS package can
% be obtained from http://gaia.fdi.ucm.es/projects/texis/
%
% Although the TeXiS template itself is distributed under the 
% conditions of the LaTeX Project Public License
% (http://www.latex-project.org/lppl.txt), the manual content
% uses the CC-BY-SA license that stays that you are free:
%
%    - to share & to copy, distribute and transmit the work
%    - to remix and to adapt the work
%
% under the following conditions:
%
%    - Attribution: you must attribute the work in the manner
%      specified by the author or licensor (but not in any way that
%      suggests that they endorse you or your use of the work).
%    - Share Alike: if you alter, transform, or build upon this
%      work, you may distribute the resulting work only under the
%      same, similar or a compatible license.
%
% The complete license is available in
% http://creativecommons.org/licenses/by-sa/3.0/legalcode
%
%---------------------------------------------------------------------

\chapter{ Discusi\'on}
\label{cap6}
\label{cap:discusion}


\begin{FraseCelebre}
\begin{Frase}
%El alma nunca piensa sin una imagen mental.
\end{Frase}
\begin{Fuente}
%Arist�teles
\end{Fuente}
\end{FraseCelebre}

\begin{resumen}
7Este cap�tulo describe todos los aspectos relacionados con las
im�genes de los documentos. En particular, describe la estructura de
directorios que \texis\ aconseja, as� como los aspectos
relacionados con la diferencia entre los formatos esperados cuando
se genera el documento final con \texttt{latex} y \texttt{pdflatex}.
\end{resumen}

%-------------------------------------------------------------------
\section{Discusi\'on}
%-------------------------------------------------------------------
\label{cap6:sec:discusion}



% Variable local para emacs, para  que encuentre el fichero maestro de
% compilaci�n y funcionen mejor algunas teclas r�pidas de AucTeX
%%%
%%% Local Variables:
%%% mode: latex
%%% TeX-master: "../ManualTeXiS.tex"
%%% End:
