
\chapter{Trabajo Individual}
\label{cap7}
\label{cap:individual}

En esta secci\'on se exponen cu\'ales han sido las contribuciones de cada uno de los miembros del equipo al proyecto.\\



%-------------------------------------------------------------------
\section{Pablo G\'omez Calvo}
%-------------------------------------------------------------------

Al tratarse de un proyecto grande, la colaboraci\'on que hemos tenido que tener ambos integrantes del grupo ha sido muy relevante para el desarrollo del proyecto. Desde que comenz\'o el proyecto hemos realizado reuniones pr\'acticamente diarias para mantenernos informados sobre la tarea que estaba realizando cada uno. \\

En los primeros momentos del desarrollo del trabajo se decidi\'o que la metodolog\'ia que se iba a utilizar era SCRUM. El uso de esta metodolog\'ia \'agil hizo que durante los primeros pasos, un integrante del grupo se ocupase de la gesti\'on. Esa tarea fue llevada a cabo por mi. Esta tarea inclu\'ia la creaci\'on y gesti\'on del software de administraci\'on de proyectos, \textit{Pivotal Tracker}.\\

En los primeros momentos del desarrollo del trabajo asum\'i la parte de gesti\'on de reuniones, creaci\'on de los repositorios, b\'usqueda de plantillas de LaTeX y  b\'usqueda de documentaci\'on que se iba necesitando en los primeros prototipos. Tambi\'en la creaci\'on de una secci\'on donde se iban almacenando enlaces de libros interesantes, revistas, \textit{papers} y todo tipo de material relacionado con el TFG.\\

Una vez que ya se ten\'ian decididos los objetivos del proyecto, realiz\'abamos sesiones de trabajo conjunto en las que se realizaron varias pruebas de concepto para mostrar a los directores del TFG. Estas pruebas de concepto formaban parte de los hitos marcados y ten\'ian como prop\'osito tener siempre una versi\'on minimamente funcional del proyecto, tal y como plasman los fundamentos de SCRUM.\\

Los prototipos en los que particip\'e fueron los siguientes:

\begin{itemize}
  \item  Implementaci\'on de la librer\'ia ZXing en Unity.
  \item Cambio de actividades en Android.
\item Lectura de un c\'odigo QR.
\item Creaci\'on de un c\'odigo QR en Unity.
\item Creaci\'on de un servidor en Unity.
\item Comunicaci\'on entre un videojuego en Unity y una aplicaci\'on Android.
\end{itemize}

A la vez que se realizaban los prototipos se utilizaban las reuniones con los directores para debatir si las funcionalidades del proyecto eran las deseadas, o si por el contrario se necesitaba hacer una revisi\'on de las funcionalidades que se le quer\'ia dar al usuario de la librer\'ia. Estas diferentes versiones del proyecto fueron, en su mayor\'ia, desarrolladas en conjunto.\\

Una vez se decidieron las funcionalidades finales que deb\'ia de tener la librer\'ia, Sergio fue el encargado de realizar la implementaci\'on final de la herramienta mientras yo me dedicaba a la redacci\'on de la memoria. Mientras la parte de redacci\'on se iba desarrollando, ambos integrantes del grupo manten\'iamos las reuniones diarias junto con un abundante n\'umero de sesiones de trabajo conjunto. En estas sesiones se pon\'ian de manifiesto los avances en la implementaci\'on de la librer\'ia para poder plasmarlos en la memoria.\\

Con el desarrollo de la librer\'ia terminado, se realizaron sesiones conjuntas para revisar las partes ya escritas y realizar las correcciones aportadas por los profesores. En paralelo a esto, se repartieron de manera equitativa las pruebas con usuarios a realizar. En sesiones conjuntas posteriores se juntaron los datos, se analizaron y fueron plasmados en la memoria.\\

Al terminar todo lo relacionado con el c\'odigo y con el proyecto, las siguientes sesiones de trabajo conjunto se dedicaron a la correci\'on de bugs y a la revisi\'on de la memoria en busca de erratas. Mientras se realizaba esta \'ultima parte, tambi\'en se mantuvieron varias sesiones con los directores del proyecto en las que se corrigieron los errores encontrados en la estructura y redacci\'on de la memoria.


%-------------------------------------------------------------------
\section{Sergio Juan Higuera Velasco}
%-------------------------------------------------------------------

Trat\'andose de un proyecto grande realizado entre dos personas y conociendo las din\'amicas de grupo gracias a proyectos anteriores, se decidi\'o que gran parte del tiempo dedicado a este proyecto iba a ser en sesiones conjuntas de trabajo. Al inicio del proyecto se realizaron sesiones conjuntas en las que buscar informaci\'on, buscar repositorios y decidir las herramientas que se iban a utilizar. Las tareas a realizar se iban almacenando en el \textit{dashboard} de tareas pendientes y se iban realizando por orden de prioridad.\\

Durante las primeras fases del proyecto se buscar\'o todo tipo de informaci\'on relacionada con el tema tratado en el TFG, con esto se pretend\'ia crear una peque\~na bibliograf\'ia con la que poder desarrollar el trabajo. Una de las primeras labores que tuvimos que desempe\~nar fue la de la b\'usqueda de libros, conferencias, art\'iculos y \textit{papers} de los temas tratados en este TFG.\\

Junto con la b\'usqueda de informaci\'on, las primeras fases del desarrollo se utilizaron para desarrollar muchos prototipos con los que experimentar las diferentes funcionalidades que se quer\'ian a\~nadir al proyecto final. Estos prototipos se realizaron en su mayor\'ia en sesiones de trabajo conjuntas. En estas sesiones conjuntas se aprovechaba para intercambiar ideas y proponer el uso de las diferentes herramientas disponibles. \\

Los prototipos en los que particip\'e fueron los siguientes:

\begin{itemize}
  \item  Cambio de actividades en Android.
  \item Comunicaci\'on entre ambos dispositivos.
\item Defininici\'on he implementaci\'on de los diferentes datagramas del protocolo de comunicaci\'on.
\item Creaci\'on de un c\'odigo QR en Unity.
\item Creaci\'on de un servidor en Unity.
\item Manejo de varios hilos concurrentes en Unity.
\item Manejo de varios hilos concurrentes en Android.
\end{itemize}

Mientras mi compa\~nero se orientaba m\'as a la b\'usqueda de informaci\'on, at\'iculos y libros relacionados con el proyecto, la labor que asum\'i fue la de investigaci\'on y aprendizaje de las diferentes herramientas que se han utilizado en la realizaci\'on de este TFG. Entre estas tareas se encuentra la de la realizaci\'on de una parte de las pruebas de concepto que se fueron realizando a lo largo del tiempo que ha durado este proyecto. Al ser conocedor de las herramientas, la parte de documentaci\'on sobre estas herramientas como Unity o Android fueron responsabilidad mia.\\

En la parte de redacci\'on la memoria, los cap\'itulos en los que m\'as he estado involucrado han sido el 3, 4 y 5. Esto se debe a que en estos cap\'itulos se explicaba todo el trabajo de desarrollo de prototipos, la posterior creaci\'on del protocolo y la implementaci\'on de la librer\'ia.

Una vez se decidieron los objetivos finales del trabajo, Pablo y yo nos dividimos un poco m\'as para poder avanzar en paralelo. Durante las sesiones de trabajo conjunto se explicaban los avances en la implementaci\'on de la librer\'ia para que mi compa\~nero pudiera plasmarlos en esta memoria. Esta fue la din\'amica que se sigui\'o hasta la realizaci\'on de las pruebas con usuarios.\\

Para las pruebas con usuarios era necesario buscar un proyecto en el que incluir la librer\'ia. Mi labor en esta parte fue la de incluir la librer\'ia en un proyecto y prepararlo para realizar una prueba de rendimiento con usuarios. Para conseguir esto implement\'e un \textit{tracker} con el que medir el rendimiento de los procesos principales de la librer\'ia. La realziaci\'on de las pruebas se repartieron a partes iguales y en sesiones posteriores se realiz\'o el an\'alisis. \\

Para el an\'alisis de los datos de las pruebas y poder sacar las gr\'aficas que se encuentran en el cap\'itulo 5, implement\'e en \textit{python} un \textit{script} con el que poder realizar el tratamiento de los datos obtenidos.\\

Como aportaci\'on final, realic\'e los siguientes  puntos de la memoria:
\begin{itemize}
  \item  Redacci\'on del ``Abstract''.
  \item Redacci\'on del cap\'itulo 1 en ingl\'es.
\item Redacci\'on de la secci\'on 6.1 ``Trabajo futuro''.
\item Redacci\'on del cap\'itulo 7.
\item B\'usqueda de im\'agenes y referencias para el cap\'itulo 2.
\end{itemize}