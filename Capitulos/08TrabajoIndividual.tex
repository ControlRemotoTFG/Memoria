
\chapter{Trabajo Individual}
\label{cap7}
\label{cap:individual}

En esta secci\'on se exponen cu\'ales han sido las contribuciones de cada uno de los miembros del equipo al proyecto.\\



%-------------------------------------------------------------------
\section{Pablo G\'omez Calvo}
%-------------------------------------------------------------------

Al tratarse de un proyecto grande, la colaboraci\'on que hemos tenido que tener ambos integrantes del grupo ha sido muy relevante para el desarrollo del proyecto. Desde que comenz\'o el proyecto hemos realizado reuniones pr\'acticamente diarias para mantenernos informados sobre la tarea que estaba realizando cada uno. \\

En los primeros momentos del desarrollo del trabajo asum\'i la parte de gesti\'on de reuniones, creaci\'on de los repositorios y b\'usqueda de plantillas de LaTeX y documentaci\'on que se iba necesitando en los primeros prototipos. En las primeras fases del proyecto mi labor fue la de buscar y almacenar a modo de bibliograf\'ia todos los art\'iculos, libros y \textit{papers} que trataban temas relacionados con el TFG.\\

Una vez ya se ten\'ian decididos los objetivos del proyecto, realiz\'abamos sesiones de trabajo conjunto en la que se realizaron varias pruebas de concepto para mostrar a los directores del TFG. Estas pruebas de concepto formaban parte de los hitos marcados y ten\'ian como prop\'osito tener una versi\'on minimamente funcional del proyecto siempre, tal y como plasman los fundamentos de SCRUM.\\

A la vez que se realizaban los prototipos se utilizaban las reuniones con los directores para debatir si las funcionalidades del proyecto eran las deseadas, o si por el contrario se necesitaba hacer una revisi\'on de las funcionalidades que se le quer\'ia dar al usuario de la librer\'ia. Estas diferentes versiones del proyecto fueron desarrolladas en conjunto.\\

Una vez se decidieron las funcionalidades finales que deb\'ia de tener la librer\'ia, Sergio fue el encargado de realizar la implementaci\'on final de la herramienta mientras yo me dedicaba a la redacci\'on de la memoria. Mientras la parte de redacci\'on se iba desarrollando, ambos integrantes del grupo manten\'iamos las reuniones diarias junto con un abundante n\'umero de sesiones de trabajo conjunto. En estas sesiones se pon\'ian de manifiesto los avances en la implementaci\'on de la librer\'ia para poder plasmarlos en la memoria.\\

Con el desarrollo de la librer\'ia terminado, se realizaron sesiones conjuntas para revisar las partes ya escritas y realizar las correcciones aportadas por los profesores. En paralelo a esto, se repartieron de manera equitativa las pruebas con usuarios a realizar. En sesiones conjuntas posteriores se juntaron los datos, se analizaron y fueron plasmados en la memoria.\\

Al terminar todo lo relacionado con el c\'odigo y con el proyecto, las siguientes sesiones de trabajo conjunto se dedicaron a la correci\'on de bugs y a la revisi\'on de la memoria en busca de erratas. Mientras se realizaba esta \'ultima parte, tambi\'en se mantuvieron varias sesiones con los directores del proyecto en las que se corrigieron los errores encontrados en la estructura y redacci\'on de la memoria.


%-------------------------------------------------------------------
\section{Sergio Juan Higuera Velasco}
%-------------------------------------------------------------------

Trat\'andose de un proyecto grande realizado entre dos personas y conociendo las din\'amicas de grupo gracias a proyectos anteriores, se decidi\'o que gran parte del tiempo dedicado a este proyecto iba a ser en sesiones conjuntas de trabajo. Al inicio del proyecto se realizaron sesiones conjuntas en las que buscar informaci\'on, buscar repositorios y decidir las herramientas que se iban a utilizar. Las tareas a realizar se iban almacenando en en el \textit{dashboard} de tareas pendientes y se iban realizando por orden de prioridad.\\

Mi labor durante las primeras fases del proyecto fue la de buscar documentaci\'on de las herramientas que se iban a utilizar y la realizaci\'on de las pruebas de concepto en sesiones conjuntas con mi compa\~nero. En estas sesiones conjuntas se aprovechaba para intercambiar ideas y proponer el uso de las diferentes herramientas disponibles. \\

Mientras mi compa\~nero se orientaba m\'as a la b\'usqueda de informaci\'on, at\'iculos y libros relacionados con el proyecto, la labor que asum\'i fue la de investigaci\'on y aprendizaje de las diferentes herramientas que se han utilizado en la realizaci\'on de este TFG. Entre estas tareas se encuentra la de la realizaci\'on de una parte de las pruebas de concepto que se fueron realizando a lo largo del tiempo que ha durado este proyecto. Al ser conocedor de las herramientas, la parte de documentaci\'on sobre estas herramientas como Unity o Android fueron responsabilidad mia.\\

Una vez se decidieron los objetivos finales del trabajo, Pablo y yo nos dividimos un poco m\'as para poder avanzar en paralelo. Durante las sesiones de trabajo conjunto se explicaban los avances en la implementaci\'on de la librer\'ia para que mi compa\~nero pudiera plasmarlos en esta memoria. Esta fue la din\'amica que se sigui\'o hasta la realizaci\'on de las pruebas con usuarios.\\

Para las pruebas con usuarios era necesario buscar un proyecto en el que incluir la librer\'ia. Mi labor en esta parte fue la de incluir la librer\'ia en un proyecto y prepararlo para realizar una prueba de rendimiento con usuarios. Para conseguir esto implement\'e un \textit{tracker} con el que medir el rendimiento de los procesos principales de la librer\'ia. La realziaci\'on de las pruebas se repartieron a partes iguales y en sesiones posteriores se realiz\'o el an\'alisis. \\

Como aportaci\'on final, en sesiones de trabajo conjunto posteriores se corrigieron bugs encontrados en el proyecto, se aplicaron las correciones propuestas por los directores y la redacci\'on de las partes de la memoria en ingl\'es. 